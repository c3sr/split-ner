% This must be in the first 5 lines to tell arXiv to use pdfLaTeX, which is strongly recommended.
\pdfoutput=1
% In particular, the hyperref package requires pdfLaTeX in order to break URLs across lines.

\documentclass[11pt]{article}

% Remove the "review" option to generate the final version.
\usepackage[review]{styles/acl}

% Standard package includes
\usepackage{times}
\usepackage{latexsym}

\usepackage{url}
\usepackage{multirow}
\usepackage{graphicx}
\usepackage{subfigure}
\usepackage{booktabs}
\usepackage{wrapfig}
\usepackage{amsmath}

% For proper rendering and hyphenation of words containing Latin characters (including in bib files)
\usepackage[T1]{fontenc}
% For Vietnamese characters
% \usepackage[T5]{fontenc}
% See https://www.latex-project.org/help/documentation/encguide.pdf for other character sets

% This assumes your files are encoded as UTF8
\usepackage[utf8]{inputenc}

% This is not strictly necessary, and may be commented out,
% but it will improve the layout of the manuscript,
% and will typically save some space.
\usepackage{microtype}


\newcommand{\comment}[1]{{\color{red} #1}}
% If the title and author information does not fit in the area allocated, uncomment the following
%
%\setlength\titlebox{<dim>}
%
% and set <dim> to something 5cm or larger.

\title{Can you ask the question again? Named entity detection via two question-answering-based classifications}

% Author information can be set in various styles:
% For several authors from the same institution:
% \author{Author 1 \and ... \and Author n \\
%         Address line \\ ... \\ Address line}
% if the names do not fit well on one line use
%         Author 1 \\ {\bf Author 2} \\ ... \\ {\bf Author n} \\
% For authors from different institutions:
% \author{Author 1 \\ Address line \\  ... \\ Address line
%         \And  ... \And
%         Author n \\ Address line \\ ... \\ Address line}
% To start a seperate ``row'' of authors use \AND, as in
% \author{Author 1 \\ Address line \\  ... \\ Address line
%         \AND
%         Author 2 \\ Address line \\ ... \\ Address line \And
%         Author 3 \\ Address line \\ ... \\ Address line}

\author{First Author \\
  Affiliation / Address line 1 \\
  Affiliation / Address line 2 \\
  Affiliation / Address line 3 \\
  \texttt{email@domain} \\\And
  Second Author \\
  Affiliation / Address line 1 \\
  Affiliation / Address line 2 \\
  Affiliation / Address line 3 \\
  \texttt{email@domain} \\}

\begin{document}
\maketitle

\begin{abstract}
Named Entity Recognition (NER) is the task of extracting informing entities belonging to predefined semantic classes from raw text. These semantic classes could be general purpose like person, location or domain-specific like genes, protein names in biomedical texts. NER has widespread applications in natural language processing (NLP) and serves as the foundation for applications like question answering, information retrieval and machine translation. Recently, the NER task has got a lot of traction in the research community with the advent of deep learning models like BERT which are able to capture textual semantics very well.

In this work, we present a detailed study approaching the NER task from three different perspectives, namely, sequence labeling, question answering (QA), and span-based classification. We propose a simple span detection and classification pipeline that first detects all mention spans irrespective of entity type and  then feeds each mention span as input to a model and expects entity type as output. This setup is the reverse of a traditional QA-based NER system where we feed entity type as input and expect mention spans as output. We also introduce explicit pattern embeddings which compliment character embeddings to learn better word representations even with less training data. Experimental results demonstrate the effectiveness of our proposed domain-agnostic techniques on multiple datasets. We set the new state-of-the-art for \texttt{BioNLP13CG} and give competitive performance on \texttt{CoNLL 2003} and \texttt{JNLPBA} datasets. Additionally, we probe into the BERT model and show that mere concatenation of external feature vectors with BERT outputs may not train effectively at the recommended low learning rates for BERT. More sophisticated feature fusion is essential.

\end{abstract}

\section{Introduction}
\label{sec:intro}
Previous approaches like sequence labeling and question answering (QA) treat the NER problem as a whole. One single model must take a sentence as input and return mention tuples with correct boundaries and correct entity type. Another possibility is to have a division of labor. We break down the NER problem into a two-step pipelined process. In the first step (\textbf{Span Detector}), we detect all mention spans in a given sentence. In the next step (\textbf{Span Classifier}), we classify these spans into their corresponding entity type. Now, we can train two separate models independently which specialize in their own sub-tasks and together solve the NER problem. We borrow the basic intuitions of QA model to solve both our sub-tasks.



\section{Pipelined NER}
\label{sec:method}
% \comment{An overall system description and architecture diagram. Details of span detection and classification go ot the subsections.}

% - question answering framework is suited to give generalized questions as input and even then expect the model to learn correct answer spans

% - QA effectiveness on identifying accurate entity boundaries (which is key for NER) is already proven by its recent success in NER

% - QA model is found to give high precision/recall for even very generalized entity types (like Genes). This motivates us to gneralize and try bringing all extractable entities under one bucket (motivation for span detection)

% (can show entity level results in appendix)

% - Entity scrambling experiment shows that QA model learns a keyword to span mapping. Then why not try to learn an exactly reverse mapping. Given entity span, associate it with a keyword.

% (show results in appendix, show that scrambing experiment shows internal term semantics of "Person", "Org" does not contribute much to NER in QA)

% Base QA Framework
% - training samples are created for each entity type for each sentence. 
% - Mention spans are treated as answers else left blank
% - Total complexity 
% - Bias towards entities with more samples. Rare entities may have very few positive samples.

% Span Detection and classification setup ensures that no extra training samples are created. Only 1 sentence for each entity mention in QA setup (which ensures optimality in terms of entity mentions now! - kind of like a lower bound)

% Next we look at span detection and classification in detail with corresponding personalizations for each

The BERT model is trained on a vast English corpus and the deeply interconnected transformer architecture ensures that the so-called foundational model learns some generalized semantics and language attributes which can be fine-tuned and reoriented to suit for a variety of NLP tasks, one such being NER.

In the NLP domain, there are several standardized problem settings or paradigms, like sequence labeling, sentence classification. An NLP application can be approached (or solved) through several of these problem settings. For example, one way of looking at NER task is through sequence labeling where each token of a sequence is to be labeled into an output class, like \texttt{Person}, \texttt{Organization}. The NER task has been traditionally looked at from this perspective. However, recently the question-answering setting has gained popularity for NER, where a question is fed as plain text, like, "Where is the Person mentioned in the text?" along with the sentence input and a model is trained to output the right span where the entity in question is present. \cite{} show the effectiveness of the question-answering paradigm for NER over the BERT architecture. Here on, we refer to this as the \texttt{BERT-QA} setup. We implemented the \texttt{BERT-QA} setup, trained it on multiple datasets (OntoNotes, BioNLP13CG, WNUT) and qualitatively studied the model outputs to draw two broad findings:

\begin{enumerate}
    \item \textbf{Finding 1}: Entity mentions belonging to the same entity type can occur in different parts of the sentence depending on sentence style.
    
    \item \textbf{Finding 2}: Entity mentions belonging to different entity types can occur in similar parts of the sentence.
\end{enumerate}

The findings are substantiated through examples in Table \ref{tab:ner_problem_ex}. Based on the findings, we can conclude that identifying entity mentions and inferring their entity types can be decoupled and treated as separate tasks. This gives the advantage that entity mention extraction rules can be shared across entity types and in case of datasets with high class imbalance, the rare entities can benefit from rules derived from mentions of the frequent ones. 

This motivates us to break down the NER task into two sub-tasks, \textit{Span Detection} and \textit{Span Classification}, each of which are trained independent of each other. \textit{Span Detection} is entity-type agnostic and forms generalized rules to identify mention spans. The \textit{Span Classification} stage takes these mention span outputs from the previous stage and associates them with their entity types.

In fact, through the similarity of our results as shown later in the paper, an NER system performing well on a dataset which does not violate the findings stated above, inherently is performing the span detection and classification tasks separately. 


\subsection{Features}
\label{sec:features}
\comment{explain the features we use}


\subsection{Span Detection}
\label{sec:span}
programatically classified the errors made by the system into some predefined categories (details in Appendix 1). From the experiments, we reveal that a shocking 80\% of errors occur primarily due to incorrect boundary detection. These are cases where the model approximately identifies the region for an entity mention in text but fails to identify the boundaries exactly. Since most NER studies report span-level (in other words, mention-level) micro-averaged F1 scores, all of these approximate mentions identified, are voided out. What causes the model to not identify these mentions exactly? The linguistic structures of several entity mentions in sentences is shared, irrespective of their entity type. However, the BER

Appendix 1:


Given a sentence $\mathcal{S}$ as a $N$-length sequence of tokens, $\mathcal{S} = \langle w_1, w_2 \ldots w_N \rangle$, the goal of this module is to output a list of spans (mention tuples) $\langle s, e\rangle$ where $s \in [1, N]$ is the \textit{start} index, $e \in [1, N]$ is the \textit{end} index. Note that here the mention tuples are not associated with an entity type. 

We formulate this as a question answering task asking the model to identify all entity spans in a given sentence. For example, the sentence, \textit{Emily}[\texttt{PERSON}] \textit{lives in United States}[\texttt{LOCATION}], is converted to the input, \textit{What is the \texttt{entity} mentioned in the text? Emily lives in United States}. This is fed to BERT model which outputs labels for each token following the \texttt{BIOE} scheme. In this example, we expect two spans, \textit{Emily} and \textit{United States}. Figure \ref{fig:span_detection} shows our span detection setup.

\begin{figure}
    \centering
    \includegraphics[width=\linewidth]{resources/span_detection}
    \caption{Span Detection Setup with \texttt{BIOE} scheme and \textit{What} as question word (colored tokens depict the generic entity type in question and gold entity mentions with expected output labels)}
    \label{fig:span_detection}
\end{figure}

- main task here is boundaries. They need to be accurate becuase the classification stage will just take the mention as input and assign a type. It will not correct the boundary later!

- also correct boundary here ensures that type classification will be better (else counter examples like, Apple - Fruit, Apple Inc - Organization) (can probably give a more realisitc example from existing used dataset)

- So, we use pattern and character embeddings (framework explanation - embedding formation, CNN usage, 50 dim) (later, have ablation on it to show its effectiveness)

- this pattetrn based formulation is a personalization over the BERT model that we could do because we split our task up into simpler sub-tasks.



\subsection{Span Classification}
\label{sec:class}
Here, we are given a sentence $\mathcal{S}$ as a $N$-length sequence of tokens, $\mathcal{S} = \langle w_1, w_2 \ldots w_N \rangle$ and a span $\langle s, e\rangle$ where $s \in [1, N]$ is the \textit{start} index, $e \in [1, N]$ is the \textit{end} index. The goal is to output a label $t$ for the span such that $t \in \mathcal{T}$, where $\mathcal{T}$ is the set of all entity types.

This is modeled as the reverse of QA model for NER described in Section \ref{sec:question_answering}. For every gold entity mention (E.g. \textit{United States}) in a training set sentence, \textit{Emily}[\texttt{PERSON}] \textit{lives in United States}[\texttt{LOCATION}], we form a sample input, \textit{Emily lives in United States. What is United States?} The sentence is fed to a BERT model where we do sequence classification. The pooled sequence embedding returned by BERT is fed to a fully connected layer and converted to a probability distribution over possible entity types. In this example, the model is expected to assign maximum probability to \texttt{LOCATION}. Figure \ref{fig:span_classification} shows our span classification setup.

\begin{figure}[h!]
    \centering
    \includegraphics[width=\linewidth]{../thesis/span_classification}
    \caption{Span Classification Setup (colored tokens depict the entity mention in question with expected output entity label)}
    \label{fig:span_classification}
\end{figure}

\subsection{Pipeline}
Both the models can be trained independently. The pipeline structure comes during the inference time. Here, every unlabeled sentence is first passed through Span Detector and for each output span, we convert to an input sample for Span Classifier.

\subsection{Salient Features}

\begin{itemize}
    \item Compared to sequence labeling and question answering approach, this span-based approach has more representative power. This is because here we have two BERT models each working on their own sub-tasks and contributing towards better NER while the other approaches just have a single model.
    
    \item Even though we are training two BERT models, they can be trained independently, in parallel. Only at inference time, we need to maintain the sequential nature.
    
    \item If we have $T$ entities of interest, then standard question answering approach creates $T$ samples for each input sentence both at train and inference time. Considering that each sentence on an average has much lesser than $T$ entity mentions, there is a lot of redundancy in this approach. 
    
    \item Our span-based approach removes QA model redundancy even though inherently we have a QA-based setup. Span Detector only sees an input sentence once and identifies all mention spans. The span classifier will work on only these identified mention spans and classify them into an entity type.
    
    \item Nevertheless, our approach has a pipeline-based structure and hence errors made by span detector propagate to the classifier. Sequence labeling and question answering approaches do not face this concern. 
    
    \item Our span-based approach shows the effectiveness of \textit{reverse question answering}. For a sentence, \textit{Emily lives in United States}, rather than asking a question of the form, \textit{"What is the \texttt{Person} mentioned in the text?"}, we ask, \textit{"What is \texttt{Emily}?"}. This opens up prospects for more intuitive forms of approaching NER, taking us closer to human understanding and interpretations.
    
    \item Comparable and even improved performance of this span-based approach compared to the general QA NER setup (results in Table \ref{tab:res_span}) shows that boundary detection of mentions has less correlation with the entity type it belongs to.
\end{itemize}

\begin{table*}[h!]
\centering
\begin{tabular}{|c|c|c|c|c|}\hline
	\textbf{} & \textbf{BioNLP13CG} & \textbf{JNLPBA} & \textbf{CoNLL 2003}\\\hline
	\texttt{Span Detection} & 90.12 & 78.35 & 95.23\\\hline
	\texttt{Span Classification} & 94.06 & 95.08 & 94.50\\\hline
	\texttt{Pipeline} & 85.89 & \textbf{75.01} & \textbf{91.64}\\\hline
	\texttt{BERT-QA} & \textbf{86.45} & 74.81 & 91.17\\\hline
	\end{tabular}
    \caption{Results: Span Pipeline (Test set Micro-F1 in \%)}
    \label{tab:res_span}
\end{table*}


\subsection{Learning Objective Function}
\label{sec:loss}
As detailed earlier, Cross Entropy is an accuracy-oriented objective while during evaluation, we calculate the F1-Score. This difference can lead to sub-optimal model training. To counteract, \cite{li2019dice} make use of S{\o}rensen-Dice coefficient(DSC)\cite{sorensen1948method, dice1945measures} and Tversky index\cite{tversky1977features} which are F-Score oriented statistics. Given sets $A$ and $B$, DSC is used to gauge similarity among two sets and is defined as,
\begin{equation}
\label{eq:dsc_set}
    DSC(A, B) = \frac{2\ \vert A \cap B \vert}{\vert A \vert + \vert B \vert}
\end{equation}
Consider $A$ as set of all positive samples predicted by a model and $B$ as set of all ground truth positives. Then, by definition of true positive ($TP$), false positive ($FP$) and false negative ($FN$) from Section \ref{sec:evaluation_metrics}, we have,
\begin{equation}
\label{eq:dsc_tp}
    TP = \vert A \cap B \vert
\end{equation}
\begin{equation}
\label{eq:dsc_a}
    \vert A \vert = TP + FP
\end{equation}'
\begin{equation}
\label{eq:dsc_b}
    \vert B \vert = TP + FN
\end{equation}
Using Equations \ref{eq:dsc_tp}, \ref{eq:dsc_a} and \ref{eq:dsc_b}, in Equation \ref{eq:dsc_set}, we get,
\begin{equation}
\label{eq:dsc_as_f1}
     DSC(A, B) = \frac{2TP}{2TP + FP + FN} = \frac{2\ \frac{TP}{TP + FN}\ \frac{TP}{TP + FP}}{\frac{TP}{TP + FN} + \frac{TP}{TP + FP}} = \frac{2\ Precision \times Recall}{Precision + Recall} = F1
\end{equation}

The dice coefficient gives equal importance to false-positives and false-negatives at training time and is more immune to data-imbalance issues\cite{sudre2017generalised, shen2018influence, kodym2018segmentation}. The above formulation (Equation \ref{eq:dsc_as_f1}) shows its equivalence to F1-score thus removing the discrepancy among training and evaluation metrics. From Equation \ref{eq:dsc_set}, for an individual sample $x_i$, dice coefficient is defined as,
\begin{equation}
\label{eq:dsc_per_sample}
    DSC(x_i) = \frac{2p_{i1}y_{i1} + \gamma}{p_{i1} + y_{i1} + \gamma}
\end{equation}
where $\gamma$ is the smoothing parameter. Then over all samples, from Equation \ref{eq:dsc_per_sample}, dice loss (DL) is defined in Equation \ref{eq:dice_loss} as:
\begin{equation}
\label{eq:dice_loss}
    DL = 1 - \frac{2\sum_i{p_{i1}y_{i1}} + \gamma}{\sum_i{p_{i1}} + \sum_i{y_{i1}} + \gamma}
\end{equation}


\section{Experimental Results}
\label{sec:exp}
\subsection{Data}
 
We validate our system using three publicly available datasets  belonging to general (OntoNotes and WNUT) and biomedical (BioNLP13CG) domains
 and a private dataset from the cybersecurity domain. 
 The cybersecurity domain data contains news articles, blogs and technical reports related to malware and vulnerabilities, and
 we denote the dataset as \texttt{CyberThreats} in this paper.
Specifically, we demonstrate the effectiveness of our algorithms for non-word, low-resource entity recognition using the datasets from multiple domains and noisy text.
The datasets include not only traditional entity types with word mentions(e.g., PERSON, LOCATION) but also many entity types with non-word, very long mentions.  
Table \ref{tab:datasets_summary} shows a summary of the datasets.
\begin{table*}[h!]
\centering
\begin{small}
\begin{tabular}{ccccrrr}\toprule
 \textbf{Dataset} & \texttt{\#Entities} & \texttt{Non-Word} & \texttt{Low Resource} & \texttt{Train} & \texttt{Dev.} & \texttt{Test} \\ \toprule 
BioNLP13CG & 16 & Yes & Yes & 3,033  & 1,003 & 1,906 \\
%JNLPBA     & 5 & Yes & No & 16,807 & 1,739 & 3,856 \\
%CoNLL2003 & 4 & No & No & 14,041 & 3,250 & 3,453 \\
CyberThreats & 8 & Yes & Yes & 38,721 & 6,322 & 9,837 \\
OntoNotes5.0 & 18 & Yes & Yes & 115,812 & 15,680 & 12,217 \\  
WNUT17 & 6 & Yes & Yes & 3,394 & 1,287 & 1,009\\
\bottomrule
\end{tabular}
\caption{Overview of the experimental datasets. \texttt{\#Entities} indicates the number of unique entity types.
\texttt{Non-Word} and \texttt{Low Resource} indicate if the dataset contains non-word mentions (e.g, mentions with digits and symbols) and
low-resource entity types respectively.  
\texttt{Train}, \texttt{Dev.} and \texttt{Test} show the number of sentences in the datasets.}
\label{tab:datasets_summary}
\end{small}
\end{table*}

\yjcomment{Shall we show the class distribution for the datasets? We need to use the names of diffrent methods consistently. Let's make a name for our system instead of Span-Pipeline.}
\comment{Different domains (entity nature \& language context) / dataset sizes / No. of entities 
- BIONLP13CG: Bio dataset (using Scibert + BERT model)
- OntoNotes: General entities / newswire (using Roberta base)
- WNUT17: emerging entities from tweets (using BERT)
Also show from result analysis that span class is simpler and gives above 90\% result where as detector is the main bottleneck}

\begin{table*}[h!]
\centering
\begin{small}
\begin{tabular}{ccccc}\toprule
 \textbf{Model} & \texttt{BioNLP13CG} & \texttt{CyberThreats} & \texttt{OntoNotes} & \texttt{WNUT17} \\ \toprule 
BERT-QA & \textcolor{red}{todo} & \textcolor{red}{todo} & \textcolor{blue}{ongoing(y4)}  & \textcolor{blue}{ongoing(y1)} \\
BERT-Span-Pipeline     & 86.70 & \textcolor{blue}{ongoing(y1)} & 90.31 & 56.30  \\
Reported SOTA & 85.56(thesis see) & N/A & 92.07(MRC-Dice) & 60.45(CL-KL)  \\
\bottomrule
\end{tabular}
\caption{Results. \texttt{\#F1} mention-level Micro F1\%. \yjcomment{let's compare with SOTA without any external data or additinonal pretraining. I consider MRC-Dice also using additional info. as they added several synonyms/hyponyms in query.   we can put SOTA without external data and with external data in separate rows if you like. We point out that our system does not rely on external knowledge which is usually unavailable for domain-speicific data and needs less computing resources than MRC}}
\label{tab:main}
\end{small}
\end{table*}

\if false
\begin{table*}[h!]
\centering
\begin{small}
\begin{tabular}{cccc}\toprule
 \textbf{Model} & \texttt{BioNLP13CG} & \texttt{OntoNotes} & \texttt{WNUT17} \\ \toprule 
BERT-Span-Pipeline     & 78.14 & 86.27(best) & 50.15  \\
Reported SOTA & N/A & 86.9(few-shot-paper-unpublished) & 50.5(CL-KL)  \\
\bottomrule
\end{tabular}
\caption{Results. (10\% setting) (may remove this table if not very important) \texttt{\#F1} mention-level Micro F1\%.}
\label{tab:main}
\end{small}
\end{table*}
\fi

\begin{table*}[h!]
\centering
\begin{small}
\begin{tabular}{ccccc}\toprule
 \textbf{Model} & \texttt{BioNLP13CG} & \texttt{CyberThreats} & \texttt{OntoNotes} & \texttt{WNUT17} \\ \toprule 
Span-Detector-CharPattern & 91.06 & 78.63 & 92.50 & 55.21  \\
Span-Detector-Vanilla     & 90.67 & \textcolor{blue}{scheduled(y1)} & \textcolor{red}{todo} & 54.85  \\
\bottomrule
\end{tabular}
\caption{Span Detector Results. \texttt{\#F1} mention-level Micro F1\%.}
\label{tab:det_ablation}
\end{small}
\end{table*}

\begin{table*}[h!]
\centering
\begin{small}
\begin{tabular}{ccccc}\toprule
 \textbf{Model} & \texttt{BioNLP13CG} & \texttt{CyberThreats} & \texttt{OntoNotes} & \texttt{WNUT17} \\ \toprule 
Span-Class-Dice & 94.27 & \textcolor{blue}{ongoing(y1)} &  96.74 & 73.40  \\
Span-Class-CE     & 94.04 & 87.58 & \textcolor{red}{todo} & 73.31  \\
\bottomrule
\end{tabular}
\caption{Span Classifier Results. \texttt{\#F1} mention-level Micro F1\%.}
\label{tab:class_ablation}
\end{small}
\end{table*}

\begin{table*}[h!]
\centering
\begin{small}
\begin{tabular}{ccccc}\toprule
 \textbf{Model} & \texttt{BioNLP13CG} & \texttt{CyberThreats(10\%)} & \texttt{OntoNotes(10\%)} & \texttt{WNUT17} \\ \toprule 
QA                & 1,372.8 & 877.1 &  7,381.8   & 568.2\\
Span-Pipeline     & 241.2 (106.3/134.9) & 145.53 (113.7/31.9) & 300.8 (185.5/115.3)  & 122.9 (98.9/24.0)\\
\bottomrule
\end{tabular}
\caption{Comparison of the training time between our method and a non-pipelined QA-based NER method. 
    The training time is reported in seconds per training epoch. The numbers in the parentheses denote the training times for span detection and span classification. }
\label{tab:train_time_ablation}
\end{small}
\end{table*}

\subsection{Experimental Setup}
We report all our results on the test sets after taking the model checkpoint corresponding to the best micro-averaged F1-score on development set. The development set evaluation takes place at steps of 0.5 training epochs. We train the models for $300$ epochs at learning rate $10^{-5}$ unless otherwise specified.

We use \texttt{transformers}\footnote{https://github.com/huggingface/transformers} python library by HuggingFace and \texttt{pytorch} for implementation and fix random seed to $42$ for replication. For general English corpora like \texttt{CoNLL 2003} and \texttt{OntoNotes 5.0}, by default, we use the pretrained \texttt{bert-base-uncased}\footnote{https://huggingface.co/bert-base-uncased} model. For biomedical datasets, \texttt{BioNLP13CG} and \texttt{JNLPBA}, we use \texttt{BioBERT-Base}\footnote{https://github.com/dmis-lab/biobert\#download} model. Note that in all our experiments, we only use the BERT-Base architecture which has around 110M trainable parameters. We use \texttt{Nvidia GeForce GTX 1080} and \texttt{Nvidia Tesla V100} gpus for model training and evaluation.

We use cross entropy loss during training unless otherwise specified. The training data is randomly shuffled and a batch size of $16$ is used with post-padding. For BERT-based models, we fix maximum sequence length to $256$ for \texttt{BioNLP13CG}, \texttt{CoNLL 2003}, \texttt{JNLPBA} datasets and $512$ for \texttt{OntoNotes 5.0} data. Unless otherwise specified, the BERT-based models output entity labels for each sub-token (as per WordPiece tokenization) of an existing token in the dataset. As a heuristic, we take the label of first sub-token as the label for the corresponding token during our evaluations.

\subsection{Baseline Systems}
\comment{Insted of BERT-xxx, people usually use the system name if it is well known (e.g., MRC) or the paper citation to denote the systems.}


\texttt{BERT}: Proposed by \cite{devlin2018bert}, BERT is a bidirectional encoder transformer\cite{vaswani2017attention}. It applies WordPiece\cite{wu2016google} tokenization on input sentence which is then passed through several encoder layers with multiple attention heads capturing sentence semantics and inter-token relationships well. The model outputs contextualized embeddings for each sub-token in the sentence. We take the last hidden layer outputs from BERT model and pass it to a fully connected layer. The outputs are converted to a probability distribution over labels space. Model parameters are initialized from a pretrained model and fine-tuned on our NER task.


\texttt{BERT-Freeze}: To understand how much semantic information is already captured in a pretrained BERT model, we use the exact same architecture as \texttt{BERT} model above but freeze the BERT model parameters. So, the only trainable parameters remain from the fully connected layer. For this setting, we use learning rate as $0.005$.

\texttt{CNN-LSTM-CRF}: \comment{better to include this as a baseline since it is very widely used}


\subsection{Performance Comparison}
Table \ref{tab:res_span} reports the results of the pipelined span detection and classification procedure and compares it with simple BERT QA setup. We present this comparison since QA model serves as the primary backbone of our span-based approach. All models here use \texttt{BIOE} tagging scheme and use \textit{What} as the question word in question formulation.

\begin{table*}[h!]
\centering
\begin{small}
\begin{tabular}{cccccc}\toprule
      &  \textbf{} & BioNLP13CG & JNLPBA & CoNLL2003 & OntoNotes5.0\\\toprule
\multirow{3}{*}{Baseline} & \texttt{BERT-Freeze} &  75.42 & 55.93 & 82.79 & 67.35 \\
                          & \texttt{BERT} & 85.99 & 74.35 & 91.36 & 83.39 \\ 
                          & \texttt{BERT-QA} & \textbf{86.45} & 74.81 & 91.17 & \\\midrule
\multirow{3}{*}{OurModel} &        \texttt{Span Detection} & 90.12 & 78.35 & 95.23 & \\
        & \texttt{Span Classification} & 94.06 & 95.08 & 94.50 & \\
        & \texttt{Pipeline} & 85.89 & \textbf{75.01} & \textbf{91.64} & \\ \bottomrule
\end{tabular}
\caption{The classificaiton results of our system and the state of the art method over 4 benchmark datasets. 
     The numbers reprent the Micro-F1 in \% on the test datasets.}
    \label{tab:res_span}
\end{small}
\end{table*}

\comment{need more results and some plots}

\if false
Next, we deep dive into the \texttt{BioNLP13CG} dataset which has $16$ entity types including several high and low-resource types. We compare the model performance at the entity type level for our 3 major NER approaches: sequence labeling, question answering and span-based pipeline. We compare our best performing model variants through entity-level and macro-averaged F1-scores. Let $\mathcal{T}$ be the set of all entity types and F1$_t$ be the F1-score for individual entity type $t \in \mathcal{T}$. Then, Macro-averaged F1-Score is defined in Equation \ref{eq:macro_f1} as:
\begin{equation}
\label{eq:macro_f1}
    \text{Macro-F1} = \frac{1}{\mathcal{\vert\mathcal{T}\vert}}\,\sum_{t\,\in\,\mathcal{T}}{\text{F1}_t}
\end{equation}
We present the comparison among the following models:
\begin{itemize}
    \item \texttt{Dice Loss}: Sequence labeling NER approach over BERT model with \texttt{BIO} tagging scheme and dice loss instead of cross entropy.

    \item \texttt{Special Symbol}: Sequence labeling NER approach over BERT with \texttt{BIO} tagging scheme and additional one-hot input feature to capture if a token is a special symbol like \textit{hyphen}, or \textit{parenthesis}.

    \item \texttt{BERT-QA (Where)}: Question answering NER approach with \texttt{BIOE} tagging scheme and \textit{Where} as the question word.

    \item \texttt{Span Based}: Pipelined approach which uses the QA setup with \texttt{BIOE} tagging scheme and \textit{What} as question word for span detection and QA-based sequence classification for span classification.
\end{itemize}

\begin{figure}
    \centering
    \includegraphics[scale=0.5]{../thesis/high_resource_entity_metrics}
    \caption{Test set Entity-level F1 scores for high resource entities in \texttt{BioNLP13CG} dataset}
    \label{fig:high_resource_entity_metrics}
\end{figure}

\begin{figure}
    \centering
    \includegraphics[scale=0.5]{../thesis/low_resource_entity_metrics}
    \caption{Test set Entity-level F1 scores for low resource entities in \texttt{BioNLP13CG} dataset}
    \label{fig:low_resource_entity_metrics}
\end{figure}

\fi


\section{Discussions}
\label{sec:discussion}
From the results of experiments reported in the previous section, we make the following observations:

\begin{itemize}
    \item From Table \ref{tab:main}, we see that \texttt{2Q-NER} outperforms sequence-tagging and QA-based baselines by a large margin on three cross-domain datasets and performs on-par with the baseline on \texttt{BioNLP13CG} corpus demonstrating the effectiveness of our proposed division of labor. 
    
    \item The results show that span detection and classification tasks have minimal correlation with each other and can be done independent of each other. 
    
    \item Compared to sequence labeling and question answering approach, \texttt{2Q-NER} has more representative power. This is because we have two BERT models each working on their own sub-tasks and contributing towards better NER while the baselines just have a single model.
    
    \item Comparison with baselines gives some additional insights into the internal functioning of BERT model which also implicitly tries to learn entity-agnostic extraction rules for mentions. Our approach explicitly models that behavior and hence is found to be more effective.
    
    \item \data{WNUT17} dataset has rare and diverse range of emerging entities crudely categorized into $6$ entity types. Training a single NER model may confuse the system to form entity-specific extraction rules. Our task segregation allows the span detection step to form generalized rules for extracting all mentions which is found to be much more effective as can be seen from Table \ref{tab:main}.
    
    \item As a sidenote, our baselines and \texttt{2Q-NER} (in Table \ref{tab:main}) outperform the previously published approaches on \data{BioNLP13CG} thus setting new state-of-the-art results. However, the credit here goes to SciBERT model, better training parameters and partially to the added character and pattern features.
    
    \item Our pipelined \texttt{2Q-NER} approach leverages the QA framework and removes its redundancies. Span Detector only sees an input sentence once and identifies all mention spans. The span classifier will work on only these identified mention spans and classify them into an entity type. Table \ref{tab:train_time_ablation} confirms and shows that the margin of improvement is more pronounced with increase in dataset size and number of entity types in the dataset.
    
    \item Table \ref{tab:train_time_ablation} considers the components of \texttt{2Q-NER} being trained sequentially. However, the approach is flexible enough to train both the BERT models independently, in parallel, reducing the train time even further. Only at inference time, we need to maintain the sequential nature.
    
    \item Comparing the F1 scores in Table \ref{tab:det_ablation} and Table \ref{tab:class_ablation}, detecting correct mention spans (span detection) is harder than disambiguating them and classifying them into entity types (span classification). Both the sub-tasks are however individually simpler than the overall NER task.
    
    \item From results in Table \ref{tab:det_ablation}, adding character and pattern features indeed helps detect better boundaries improving precision with comparable recall thus leading to a higher F1 score in \textit{Span Detection Module}.
    
    \item From results in Table \ref{tab:feature_ablation}, adding character and patterns individually has a positive impact on overall span detection performance. Both together give an even better performance. Part-of-speech semantics is well captured by the existing BERT features and the added character and pattern features and hence does not need to be explicitly learnt.
    
    \item From results in Table \ref{tab:class_ablation}, we see that \texttt{Dice Loss} helps handle the class imbalance issues seen in \textit{Span Classification Module} and gives a slight performance improvement over \texttt{Cross Entropy Loss}.
\end{itemize}

\subsection{Qualitative Analysis}
In Table \ref{tab:quality} we show some sample predictions made by our \texttt{2Q-NER} system and compare them with those made by the corresponding \texttt{BERT-QA} system. 
\begin{itemize}
    \item \texttt{2Q-NER} is better in detecting emerging entities and out-of-vocabulary terms (like new movies, softwares) which may be rare in the dataset and have high diversity. This can be attributed to \textit{Span Detection Module} being stronger in generalizing and sharing entity extraction rules across multiple entity types.
    
    \item \texttt{BERT-QA} gets confused in cases where entities have special symbols within them (like hyphens in scientific terms, comma in some multi-word entities). Character and pattern features in \textit{Span Detection Module} help handle such cases well.
    
    \item \texttt{BERT-QA} model develops a bias towards more common entity types like \texttt{Location} and may misclassify rare entity mentions (like \texttt{Products}) when they occur in a similar context. \texttt{2Q-NER} handles such cases well with a dedicated \textit{Span Classification Module} designed to use dice loss.
\end{itemize}

\begin{table*}[h!]
\centering
\begin{small}
\begin{tabular}{ccc}\toprule
Category & Model & Example \\\toprule
\multirow{2}{*}{General Detection} & \texttt{BERT-QA} & \textit{CVS selling their own version of ...} \\
    & \texttt{2Q-NER} & \textit{\textcolor{blue}{CVS}[\texttt{Corporation}] selling their own version of ...} \\ \midrule
\multirow{2}{*}{Emerging Entities} & \texttt{BERT-QA} & \textit{Does Rogue One create a plot hole in Return of the Jedi ... } \\
    & \texttt{2Q-NER} & \textit{Does \textcolor{blue}{Rogue One}[\texttt{Creative Work}] create a plot hole in \textcolor{blue}{Return of the Jedi}[\texttt{Creative Work}] ... } \\ \midrule
\multirow{2}{*}{Scientific Terms} & \texttt{BERT-QA} & \textit{The MVD and Ki - 67 LI were higher ... } \\
    & \texttt{2Q-NER} & \textit{The MVD and \textcolor{blue}{Ki - 67}[\texttt{Gene}] LI were higher ...} \\ \midrule
\multirow{2}{*}{Boundaries} & \texttt{BERT-QA} & \textit{.. Hotel Housekeepers Needed in Spring , \textcolor{blue}{TX}[\texttt{Location}] ... } \\
    & \texttt{2Q-NER} & \textit{.. Hotel Housekeepers Needed in \textcolor{blue}{Spring , TX}[\texttt{Location}] ... } \\ \midrule
\multirow{2}{*}{Out of Vocab Terms} & \texttt{BERT-QA} & \textit{Store SQL database credentials in a webserver} \\
    & \texttt{2Q-NER} & \textit{Store \textcolor{blue}{SQL}[\texttt{Product}] database credentials in a webserver} \\ \midrule
\multirow{2}{*}{Entity Classification} & \texttt{BERT-QA} & \textit{Why do so many kids in \textcolor{blue}{Digimon}[\texttt{Location}] wear gloves?} \\
    & \texttt{2Q-NER} & \textit{Why do so many kids in \textcolor{blue}{Digimon}[\texttt{Product}] wear gloves?} \\ \bottomrule
\end{tabular}
\caption{Examples from multiple datasets comparing performance of \texttt{2Q-NER} and \texttt{BERT-QA} systems}
    \label{tab:quality}
\end{small}
\end{table*}

\section{Related Work}
\label{sec:related}
In the past few years, deep learning approaches have been increasingly applied for NER \cite{torfi2020natural, li2020survey}, a popular architecture being CNN-LSTM-CRF\cite{ma2016end}. 
% Training complex deep learning systems have however been limited by the shortage of labeled training data. Nevertheless, there have been efforts to generate noisy labeled data as weak supervision signals for training. \cite{shang2018learning} propose AutoNER framework which uses distant supervision, \cite{arora2017extracting} use regular expression patterns for artificial training data generation, \cite{zhou2019dual} propose adversarial perturbations and \cite{liu2018empower, liu2018efficient} train a task-aware language model from unlabeled data which guides NER. 
Recently with the advent of BERT\cite{devlin2019bert}, it has become possible to transfer knowledge from massive pretrained language models and fine-tune them giving good performance even on small datasets.

\cite{xu2021better} propose a Syn-LSTM setup leveraging dependency tree structure along with pretrained BERT embeddings for NER. \cite{li2020MRC} and \cite{li2019dice} have a system very similar to our system and use QA-based setup over BERT for NER. However, they enumerate all possible span start and end locations in a sentence leading to an $\mathcal{O}(n^2)$ complexity where $n$ denotes the sentence size. Our architecture design does a token level classification and hence has $\mathcal{O}(n)$ complexity. Recently, \cite{yan2021unified} propose a generative framework leveraging BART\cite{lewis2019bart} for NER. \cite{yu2020named} propose a biaffine model utilizing pretrained BERT and FastText \cite{bojanowski2017enriching} embeddings along with character-level CNN setup over a bi-LSTM architecture. All of these models leverage BERT in some way or the other and report good performance on \data{OntoNotes 5.0} corpus however all of them make use of BERT-Large architecture. 

\cite{wang2021improving} give good performance on \data{WNUT17} dataset by leveraging external knowledge and a cooperative learning setup. Recently, \cite{nguyen2020bertweet} propose a new BERT model, BERTweet trained on English tweets. Currently, we use the standard pretrained BERT model by \cite{devlin2019bert} on \data{WNUT17}. Our framework can easily be trained using BERTweet model as the backbone and this can enhance our results on \data{WNUT17} even further.

In scientific and biomedical domain, NER has its own set of challenges. Entity mentions are long and may have alpha-numeric symbols and chemical formulas which may be hard for a language model to make sense of. On \data{BioNLP13CG} corpus, \cite{crichton2017neural} report $78.90$ as test set micro-F1 in a multi-task learning setup and \cite{neumann2019scispacy} report $77.60$ using their SciSpacy system. BioBERT\cite{lee2020biobert} authors does not state their performance on this dataset however \cite{banerjee2019knowledge} replicate the BioBERT model as a baseline and report $85.56$. Other works like \cite{wang2019cross} use a multi-task learning framework and combine labelled data from multiple corpora mapping entity tags across corpora to coherent classes. \cite{wang2019distantly} make use of a setup similar to AutoNER\cite{shang2018learning} tailor-made for biomedical NER. 


\section{Conclusion}
In this work, we demonstrate that NER can be broken down into a pipeline of two simpler sub-tasks, span detection followed by span classification. Models can be trained for the sub-tasks independent of each other. Using the QA-framework over the BERT architecture, we show that this division of labor is not only effective but also significantly efficient, through experiments on multiple cross-domain datasets. Nevertheless, our approach has a pipeline-based structure and hence errors made during span detection propagate to the span classification stage. Sequence labeling and question answering approaches do not face this concern. However, our proposed breakdown does gives more flexibility to develop models and cater to the needs of the individual sub-tasks more effectively. We show that explicitly modeling character and pattern features helps detect better boundaries and dice loss helps handle class imbalance issues.

Through this paper, we open up the possibility of breaking down complex tasks into smaller sub-tasks and fine-tuning large pretrained language models for them individually. Interestingly, our span classification setup is the reverse of a standard QA-setup for NER. For a sentence, \textit{Emily lives in United States}, rather than asking a question of the form, \textit{``What is the \texttt{person} in the text?"} as done in NER, we ask \textit{``What is \texttt{Emily}?"}. This opens up prospects for more intuitive and creative forms of approaching NER and shows the power of BERT model yet again in capturing the semantics and query intent well. During our qualitative study, we identify that boundary detection serves as a primary bottleneck in NER performance and encourage the research community to design architectures or new training objectives tailor-made to handle mention boundaries more effectively. 
    
Currently, in our \textit{Span Detection Module}, all required entity mentions are being grouped into a single class. As a potential future work, we expect to get even better performance by a hierarchical extension of our setup, like a decision tree. At the top level, we detect mentions belonging to some crude categories and gradually break them down into more fine-grained buckets. These buckets can be designed to maximize information gain using some measure of entropy / similarity of mentions being grouped together.


%\section*{Acknowledgements}

% Entries for the entire Anthology, followed by custom entries
\bibliographystyle{styles/acl_natbib}
\bibliography{thesisrefs.bib}
%\bibliography{anthology,custom}

%\appendix

%\section{Example Appendix}
%\label{sec:appendix}


\end{document}
