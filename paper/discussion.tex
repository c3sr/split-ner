\subsection{Observations}
\begin{itemize}
    \item \textbf{Span Detection}: Detecting all mention spans together without classification is a simpler problem for the model than full NER and we get better performance on this sub-task compared to complete NER task in QA setup.
    
    \item \textbf{Span Classification}: Given that spans are pre-identified, classifying them to an entity type is a relatively simple task for the BERT model. On all datasets, we see around $95\%$ Micro-F1 on test set.
    
    \item \textbf{Pipeline}: The pipelined procedure gives comparable and even better performance than standard QA NER model on all datasets demonstrating the effectiveness of this division of labor. 
    
    \item Since out \texttt{Pipeline} results are comparable to \texttt{BERT-QA} model, we conclude that internally \texttt{BERT-QA} model also tries to logically segregate boundary detection and classification as separate tasks.
    
    \item The results of span pipeline are limited by the performance of the span detector part. Since this procedure is pipelined, errors in this first step propagate to the next step. Boundary detection serves as the primary challenge in Span Detector and has a large scope for improvement on biomedical datasets.
    
    \item Qualitative analysis reveals that both \texttt{BERT-QA} and \texttt{Span Detector} share very similar boundary detection issues.
\end{itemize}

\subsection{Salient Features}

\begin{itemize}
    \item Compared to sequence labeling and question answering approach, this span-based approach has more representative power. This is because here we have two BERT models each working on their own sub-tasks and contributing towards better NER while the other approaches just have a single model.
    
    \item Even though we are training two BERT models, they can be trained independently, in parallel. Only at inference time, we need to maintain the sequential nature.
    
    \item If we have $T$ entities of interest, then standard question answering approach creates $T$ samples for each input sentence both at train and inference time. Considering that each sentence on an average has much lesser than $T$ entity mentions, there is a lot of redundancy in this approach. 
    
    \item Our span-based approach removes QA model redundancy even though inherently we have a QA-based setup. Span Detector only sees an input sentence once and identifies all mention spans. The span classifier will work on only these identified mention spans and classify them into an entity type.
    
    \item Nevertheless, our approach has a pipeline-based structure and hence errors made by span detector propagate to the classifier. Sequence labeling and question answering approaches do not face this concern. 
    
    \item Our span-based approach shows the effectiveness of \textit{reverse question answering}. For a sentence, \textit{Emily lives in United States}, rather than asking a question of the form, \textit{"What is the \texttt{Person} mentioned in the text?"}, we ask, \textit{"What is \texttt{Emily}?"}. This opens up prospects for more intuitive forms of approaching NER, taking us closer to human understanding and interpretations.
    
    \item Comparable and even improved performance of this span-based approach compared to the general QA NER setup (results in Table \ref{tab:res_span}) shows that boundary detection of mentions has less correlation with the entity type it belongs to.
\end{itemize}
