% \documentclass[draftthesis,tocnosub,noragright,centerchapter,fullpagesingle,12pt]{uiuc_csthesis18}
\documentclass[tocnosub,noragright,centerchapter,fullpagesingle,12pt]{uiuc_csthesis18}

% Updated version of the ECE department's latex resources

% Use draftthesis for notes and date markings on every page.  Useful when you
%   have multiple copies floating around.
% Use offcenter for the extra .5 inch on the left side. Needed with fullpage and fancy.
% Use mixcasechap for compatibility with hyperref package, which does NOT like all caps default
% Use edeposit for the adviser/committee on the title page.
% Use tocnosub to suppress subsection and lower entries in the TOC.
% PhD candidates use "proquest" for the proquest abstract.

\makeatletter

\usepackage{setspace}  % Useful for single, 1.5, and double spacing
\usepackage[numbers, sort]{natbib}  % Useful for formatting reference section
\usepackage{url}  % Useful for URLs
%\usepackage{hyperref}  % Another package useful for URLs
\usepackage{subcaption}

\usepackage{lscape}  % Useful for wide tables or figures.
% Following command definition is from Stack Exchange: https://tex.stackexchange.com/questions/278113/single-landscape-page-with-page-number-at-the-bottom 
% It adds *rotated* page numbers to the bottom of landscaped pages to meet the Graduate College standards (see page 7 here: https://grad.illinois.edu/files/pdfs/thesis-sample-chapter-straight-numbering.pdf)
\def\fillandplacepagenumber{
	\par
	\pagestyle{empty}
	\vbox to 0pt{\vss}
	\vfill
	\vbox to 0pt{
		\baselineskip 0pt
		\hbox to \linewidth{\hss}
		\baselineskip\footskip
		\hbox to \linewidth{\hfil\thepage\hfil}\vss
	}
}

%%%%%%%%%%%%%%%%%%%%%%%%%%%%%%%%%%%%%%%%%%%%%%%%%%%%%%%%%%%%%%%%%%%%%%%%%%%%%%%
% FIGURE PACKAGES
%
\usepackage{graphicx}  % Please import figures that are *high resolution* PDFs
%\usepackage{epsfig}   % or EPS files
\usepackage{caption}
%\usepackage{subfigure}  % Useful for subfigures
%\usepackage{subcaption}  % Useful for captioning subfigures
%%%%%%%%%%%%%%%%%%%%%%%%%%%%%%%%%%%%%%%%%%%%%%%%%%%%%%%%%%%%%%%%%%%%%%%%%%%%%%%
% TABLE PACKAGES
%
\usepackage{booktabs}  % Useful for high quality tables (e.g., you can replace \hrule with \toprule, \midrule, and \bottomrule).
%\usepackage{multicol}
%\usepackage{multirow}
%%%%%%%%%%%%%%%%%%%%%%%%%%%%%%%%%%%%%%%%%%%%%%%%%%%%%%%%%%%%%%%%%%%%%%%%%%%%%%%
% MATH PACKAGES (Comment out this section if unnecessary for your dissertation)
%
\usepackage{amsfonts}
\usepackage{amsmath}
\usepackage{amssymb}
\usepackage{amstext}
\usepackage{amsthm}

% Change numbering of definitions, lemmas, theorems, etc to meet the Graduate College standards
\theoremstyle{definition}
\newtheorem{definition}{Definition}[chapter]
\newtheorem{lemma}{Lemma}[chapter]
\newtheorem{theorem}{Theorem}[chapter]
\newtheorem{corollary}{Corollary}[chapter]
\newtheorem{conjecture}{Conjecture}[chapter]
\newtheorem{remark}{Remark}[chapter]

\renewcommand{\qedsymbol}{QED.}  % Change symbol at end of proofs to meet the Graduate College standard
%%%%%%%%%%%%%%%%%%%%%%%%%%%%%%%%%%%%%%%%%%%%%%%%%%%%%%%%%%%%%%%%%%%%%%%%%%%%%%%
% ALGORITHM AND CODE PACKAGES (Comment out this section if unnecessary for your dissertation)
%
\usepackage{listings}  % Useful for formatting code blocks, see here for further information about formatting code: https://en.wikibooks.org/wiki/LaTeX/Source_Code_Listings
\usepackage[ruled]{algorithm2e}  % Useful for formatting algorithms (pseudocode)
\numberwithin{algocf}{chapter}     % Change numbering of algorithms to meet the Graduate College standards

%%%%%%%%%%%%%%%%%%%%%%%%%%%%%%%%%%%%%%%%%%%%%%%%%%%%%%%%%%%%%%%%%%%%%%%%%%%%%%%
% COVERPAGE
%

% Uncomment the appropriate one of the following four lines:
\msthesis
% \phdthesis
%\otherdoctorate[abbrev]{Title of Degree}
%\othermasters[abbrev]{Title of Degree}

\title{Domain-Agnostic Named Entity Recognition on Unstructured Text}
\author{Jatin Arora}
\department{Computer Science}
\degreeyear{2021}

% Advisor name is required for
% - doctoral students for the ProQuest abstract
% - master's students who do not have a master's committee
\advisor{Professor Jiawei Han}

% Uncomment the \committee command for
% - all doctoral students
% - master's students who have a master's committee
% \committee{Professor Firstname Lastname, Chair\\
%         Professor Firstname Lastname} % etc.

\begin{document}

%%%%%%%%%%%%%%%%%%%%%%%%%%%%%%%%%%%%%%%%%%%%%%%%%%%%%%%%%%%%%%%%%%%%%%%%%%%%%%%
% COPYRIGHT
%
\copyrightpage
\blankpage

%%%%%%%%%%%%%%%%%%%%%%%%%%%%%%%%%%%%%%%%%%%%%%%%%%%%%%%%%%%%%%%%%%%%%%%%%%%%%%%
% TITLE
%
\maketitle

%\raggedright
\parindent 1em%

\frontmatter

%%%%%%%%%%%%%%%%%%%%%%%%%%%%%%%%%%%%%%%%%%%%%%%%%%%%%%%%%%%%%%%%%%%%%%%%%%%%%%%
% ABSTRACT
%
\begin{abstract}
Named Entity Recognition (NER) is the task of extracting informing entities belonging to predefined semantic classes from raw text. These semantic classes could be general purpose like person, location or domain-specific like genes, protein names in biomedical texts. NER has widespread applications in natural language processing (NLP) and serves as the foundation for applications like question answering, information retrieval and machine translation. Recently, the NER task has got a lot of traction in the research community with the advent of deep learning models like BERT which are able to capture textual semantics very well.

In this work, we present a detailed study approaching the NER task from three different perspectives, namely, sequence labeling, question answering (QA), and span-based classification. We propose a simple span detection and classification pipeline that first detects all mention spans irrespective of entity type and  then feeds each mention span as input to a model and expects entity type as output. This setup is the reverse of a traditional QA-based NER system where we feed entity type as input and expect mention spans as output. We also introduce explicit pattern embeddings which compliment character embeddings to learn better word representations even with less training data. Experimental results demonstrate the effectiveness of our proposed domain-agnostic techniques on multiple datasets. We set the new state-of-the-art for \texttt{BioNLP13CG} and give competitive performance on \texttt{CoNLL 2003} and \texttt{JNLPBA} datasets. Additionally, we probe into the BERT model and show that mere concatenation of external feature vectors with BERT outputs may not train effectively at the recommended low learning rates for BERT. More sophisticated feature fusion is essential.
  % Put the text for the abstract in a file called "abstract.tex", and it will be inserted here.
\end{abstract}

%%%%%%%%%%%%%%%%%%%%%%%%%%%%%%%%%%%%%%%%%%%%%%%%%%%%%%%%%%%%%%%%%%%%%%%%%%%%%%%
% DEDICATION (Uncomment this section if desired)
%
\begin{dedication}
To my parents, for their love and support.
% Whatever dedication you want, for example: "To my parents, for their love and support."
\end{dedication}

%%%%%%%%%%%%%%%%%%%%%%%%%%%%%%%%%%%%%%%%%%%%%%%%%%%%%%%%%%%%%%%%%%%%%%%%%%%%%%%
% ACKNOWLEDGMENTS
%
\begin{acknowledgments}
First and foremost, I would like to thank my advisor, Prof. Jiawei Han for his continuous guidance and support. Prof. Han's teachings, his principles and dedication to work has always been a source of inspiration for me and has encouraged me to pursue my research interests relentlessly. With Prof. Han, I can confidently say that I have been treated not just as his student, but as his family. His care extends far beyond the call of duty and I am fortunate to have got a chance of being his student.

I would also like to thank all the members of our Data Mining Group. We are one big family and it has been a pleasure to work with such a talented group of peers and share ideas. Especially, I would like to thank Xuan Wang and Jiaming Shen for their guidance throughout my course of study.

I am also very grateful to Dr. Youngja Park, my mentor from IBM Research, without whom this work would not have been possible. Thank you so much Youngja for always being there for me.

Besides, I thank all my friends, well wishers and family members for always having faith in me and for their motivation and support in times of need.  % Inserts contents from "acknowledgements.tex" here
\end{acknowledgments}

%%%%%%%%%%%%%%%%%%%%%%%%%%%%%%%%%%%%%%%%%%%%%%%%%%%%%%%%%%%%%%%%%%%%%%%%%%%%%%%
% TABLE OF CONTENTS
%
\tableofcontents

\mainmatter

%%%%%%%%%%%%%%%%%%%%%%%%%%%%%%%%%%%%%%%%%%%%%%%%%%%%%%%%%%%%%%%%%%%%%%%%%%%%%%%
% INSERT REAL CONTENT HERE
%

\chapter{Introduction}
\label{chp:intro}
% Motivations for IE
As quoted by Forbes\footnote{https://www.forbes.com/sites/bernardmarr/2018/05/21/how-much-data-do-we-create-every-day-the-mind-blowing-stats-everyone-should-read/?sh=7cf99c3f60ba}, 2.5 quintillion bytes of new data is generated every day and this number is fast growing. In fact, 90 percent of all the data has been generated just in the last 2 years. Quite rightly said, it is indeed a \textit{data-driven world}, and the future is going to be no different. This vast multitude of raw data is a gold mine with an abundance of knowledge hidden inside it. A lot of it is in the form of unstructured text including books, research papers, news articles, blog posts, customer reviews, tweets etc. However, the pace of data growth has made it humanly impossible to manually browse through everything and get the required knowledge thus motivating research towards automated information extraction and knowledge discovery. 

In essence, the ultimate goal is to acquire knowledge from raw data which in-turn helps guide decision making. This can be viewed as a step-by-step process in which the  first step is to parse large volumes of raw text and extract informing entities along with their inter-relationships. Next is to organize this in the form of a knowledge graph preserving the interconnections. Finally, given a plain text query from the user, convert it into some graph operations to retrieve and return the relevant results.

In this thesis we focus on the first step i.e. information extraction (IE) which itself covers a diverse range of tasks. Named Entity Recognition (NER) deals with the extraction of important entities of interest from text. Relation extraction is the process of extracting inter-relationships among informing entities. Sentiment analysis deals with classifying the overall sentiment conveyed in given text. Question answering is the study of extracting an answer for a given question from a given input text. We primarily focus on the named entity recognition task here and next look at some of its applications.

% applications of IE (industry documents)
\textbf{Industry Documents}. Most modern-day organizations have to deal with lots of documents. These include annual reports, purchase invoices, salary slips, client contracts, resumes, emails etc. This makes up a vast and diverse pool of content-rich data which is persisted from a regulatory perspective and otherwise. Named entity recognition on this data can help draw important insights from past decisions and further improve the current business model. However, many such documents contain protected and sensitive information interspersed within. For example, employee records contain their address, date of birth, phone number, social security number (SSN) etc. which is private information protected by laws like CCPA(California Consumer Privacy Act) etc. Another application of NER, from an information security perspective, is to identify and redact such content and prevent sensitive information leakage.

% applications of IE (scientific literature)
\textbf{Scientific Literature}. The research community is ever flourishing, and it has become increasingly difficult for any researcher to remain up to date with all the latest developments even in their domain itself. As quoted in the first ever machine generated book of Chemistry published in Springer Nature\cite{writer2019lithium}, even in the niche research domain of Li-ion batteries, there were more than 53,000 research papers released in just the past 3 years. It is hence essential to have information extraction and summarization tools to filter out important content and NER becomes a vital step in this process.

% applications of IE (web data)
\textbf{Web Data}. In this age of the internet, the amount of online content has been exploding. There are around 500 million tweets everyday\footnote{https://www.internetlivestats.com/twitter-statistics/}. E-commerce has been blooming on websites like Amazon\footnote{https://www.amazon.com/}. According to some stats\footnote{https://landingcube.com/amazon-statistics/}, Amazon ships around 1.6 million packages per day. After making purchases, people voice their opinions and compare products and their features by writing reviews. Around 79\% of customers check posted reviews before making a purchase. These vast volumes of reviews have lots of valuable information embedded including product names, their features, opinion predicates etc. and is a trending NER application.

NER has been a popular research area for long and has seen transitions in parallel, with advances in the field of machine learning. Broadly, NER techniques can be categorized into rule-based, unsupervised, feature engineered and more recently, deep learning-based techniques. For NER, it is important to be able to understand the semantics of the sentence and how different entities are woven together by the rules of grammar and interacting with each other. In recent years, deep learning techniques like recurrent neural networks, LSTMs\cite{hochreiter1997long} and now transformer-based\cite{vaswani2017attention} architectures like BERT\cite{devlin2018bert} have made great progress in language modeling and capturing these inter-token dependencies which has aided the performance of named entity recognition. 

% However, there are still some open challenges to address:
% \begin{itemize}
%     \item Language dynamics drastically changes between general English news text and content-heavy scientific research literature. This also has impact on corresponding NER systems.
    
%     \item Deep learning approaches require a large amount of labeled training data. However, several entities requiring extraction do not have enough labeled samples making it difficult to extract them. (We handle this by pattern modeling)(We also hypothesize that information of high-resource entities facilitates the extraction of low-resource entities in its vicinity)
    
%     \item Different entities have different semantics which may require explicit modeling. Entities like \texttt{Person}, \texttt{Location} are wordy, while others like \texttt{Date}, \texttt{SSN}, \texttt{Telephone} can be numeric codes having patterns within them. (We handle this by parallel Char-CNN+Patterns)
% \end{itemize}

In this work, we use the popular \texttt{CNN-LSTM-CRF} and BERT models as baselines and develop on top of them. We present a thorough study and compare and contrast the NER task from three different perspectives, as a sequence labeling problem, a question answering-based approach and span detection and classification task. The contributions of our work are three-fold:

\begin{itemize}
    \item We introduce a novel pattern modeling approach which converts sparse character space to dense pattern space for more effective training of alphanumeric entities even with minimal training samples.
    
    \item Our system achieves new state-of-the-art on \texttt{BIONLP13CG} dataset which consists of 16 fine-grained alphanumeric named entity types and achieves competitive results in both biomedical as well as general English news domains.
    
    \item We show that mere concatenation of additional semantic features with BERT (transformer-based) features is not able to achieve its full modeling potential at the recommended low learning rates. More sophisticated feature fusion is essential.
\end{itemize}

% In this work, we focus on latest present a detailed NER analysis across different domains including news articles and biomedical literature. We present a general domain-agnostic model architecture for NER and also present detailed ablations of the model components. Additionally, we do a thorough qualitative analysis of the used corpora and present the bottlenecks of existing benchmarks as well as concrete scopes of improvement to give direction to future research.  % Inserts content from "introduction.tex" here

\chapter{Preliminaries}
\label{chp:preliminaries}
\section{Nature of Entities}
\label{sec:nature_of_entities}
Language dynamics and style of writing vary across domains and so does the nature of informing entities in corresponding sentences. Some characteristics of entities include:

\begin{itemize}
    \item \textbf{Multi-word vs single-word}. Common approach to handle such cases is using \texttt{BIO}, \texttt{BIOE} tagging schemes to learn entity boundaries.
    
    \item \textbf{Alpha-numeric vs alphabetic words vs abbreviations}. Some entities like \textit{passport number}, \textit{telephone number} etc. have semantics captured through intrinsic alpha-numeric patterns. \textit{Locations}, \textit{person name} etc. and abbreviation have some common terms or external gazetteers or surrounding words capturing their semantics.
    
    \item \textbf{Broad vs narrow spectrum}. Some entity classes may be broad with many diverse examples and intricacies, while others may be specific and have a narrow spectrum which is simpler to capture.
    
    \item \textbf{Nested entities}. Related to diversity, it may be possible that some specific entities are a sub-class of a more general entity. In other words, there may be a hierarchy/taxonomy of entities to consider. For example, \textit{U.S. Embassy in India} is a \texttt{Facility} where \textit{U.S.} and \textit{India} are \texttt{Locations}. Such examples can be found in biomedical domain in GENIA corpus\cite{} and news article domain in ACE-2004\cite{} and ACE-2005\cite{} corpora. \cite{finkel2009nested} align sentences into a tree-structure with entities for handling this and \cite{li2019unified} extract each entity separately by querying a sentence multiple times in a question-answering framework.
    
    \item \textbf{High vs low-resource}. Depending on importance to the underlying use-case and domain, not all labeled entities may have an equable representation in the data. Some may have more examples (\textit{high-resource entities}) while others may have only a few (\textit{low-resource entities}). 
    
\end{itemize}

Based on the above factors, our goal is to develop an approach for named entity recognition that is domain-agnostic and which does not develop biases towards commonly seen entities and learns a meaningful representation for low-resource entities as well. Hence, we conduct our experiments on multiple datasets as detailed below. 

\section{Evaluation Metrics}
% arorja: TODO

\section{Tagging Scheme}
\label{sec:tagging_scheme}
In order to capture the boundaries of entity mentions correctly, it is a standard practice to label each token as per a tagging scheme like \texttt{SingleTag}, \texttt{BIO}, \texttt{BIOE}, \texttt{BIOES} as detailed below.

\begin{itemize}
    \item \texttt{SingleTag}: Tag each token of a sentence with class label like \texttt{PERSON}, \texttt{ORG} etc. if it is a part of a known entity or \texttt{O} (or \texttt{NONE}) tag if it does not belong to any required entity.
    
    \item \texttt{BIO}: This tries to capture the boundaries of entities more explicitly than \texttt{SingleTag}. Use \texttt{B-Tag} (read \textit{Begin Tag}) for first token in the entity marked with \texttt{Tag} and all continuing tokens are marked with \texttt{I-Tag} (read \textit{Intermediate Tag}). If a token does not belong to any entity, it is classified as \texttt{O}.
    
    \item \texttt{BIOE}: Apart from the \texttt{BIO} tags, here the ending token of an entity is tagged with \texttt{E-Tag} (read \textit{End Tag}). Unigram entities are tagged with \texttt{B-Tag}.
    
    \item \texttt{BIOES}: Additionally over \texttt{BIOE} scheme, unigram entities are tagged with special \texttt{S-Tag} (read \textit{Single Tag}).
\end{itemize}

\section{Datasets}

Table \ref{tab:datasets_summary} gives a summary of the datasets and nature of entities they posses. For each dataset we work with standard \texttt{Train}/\texttt{Dev}/\texttt{Test} splits as used by previous studies. Our datasets are persisted in files in tab-separated (\texttt{TSV}) format. Each token in a sentence corresponds to a row with 4 tab-separated entries (token, part-of-speech tag, dependency-parse tag, output entity tag). Two sentences are separated by an empty line in the file. We follow the \texttt{BIO} tagging scheme in the persisted data files.

\begin{table}[h!]
	\begin{tabular}{|c|c|c|p{5em}|p{6em}|c|}\hline
	\textbf{Dataset} & \textbf{Source} & \textbf{\#Entities} & \textbf{Alpha-numerics} & \textbf{High/Low Resource} & \textbf{Structure}\\\hline
	\texttt{CoNLL 2003} & News & 4 & No & High & Flat\\\hline
	\texttt{OntoNotes 5.0} & News & 18 & No & High \& Low & Flat\\\hline
	\texttt{BIONLP13CG} & Biomedical & 16 & Yes & High \& Low & Nested\\\hline
	\texttt{JNLPBA} & Biomedical & 4 & Yes & High & Flat\\\hline
	\end{tabular}
	\caption{Datasets Summary}
	\label{tab:datasets_summary}
\end{table}

\subsection{CoNLL 2003 Dataset}

The \texttt{CoNLL 2003}\cite{sang2003introduction} corpus is a collection of news wire articles from the Reuters corpus and has been manually annotated with 4 classes, \texttt{PER} (Person), \texttt{ORG} (Organization), \texttt{LOC} (Location) and \texttt{MISC} (Miscellaneous) entities which do not belong to the other 3 classes. We obtain the dataset from \texttt{datasets}\footnote{https://huggingface.co/datasets/conll2003} package. The \texttt{MISC} class can be considered more diverse than the other 3 classes. Alphanumeric entity tokens only constitute \texttt{0.05\%} of all entity tokens and hence all named entities are predominantly non-alphanumeric. The entities are flatly labelled i.e do not have overlaps or nesting structure. As per Table \ref{tab:conll_entity_distribution}, none of the entities are low-resource. Average sentence length is \texttt{14.53} tokens. Table \ref{tab:conll_dataset_split} shows the \texttt{Train}/\texttt{Dev}/\texttt{Test} split. 

\begin{table}[h!]
\begin{subtable}[t]{.48\linewidth}
\centering
\begin{tabular}{|c|c|}\hline
	\textbf{Entity} & \textbf{Count}\\\hline
	\texttt{PER} & 10059\\\hline
	\texttt{ORG} & 9323\\\hline
	\texttt{LOC} & 10645\\\hline
	\texttt{MISC} & 5062\\\hline
	\end{tabular}
	\caption{Entity Distribution}
	\label{tab:conll_entity_distribution}
% }
\end{subtable}
\begin{subtable}[t]{.48\linewidth}
\centering
\begin{tabular}{|c|c|}\hline
	\textbf{Split} & \textbf{\# Sentences}\\\hline
	\texttt{Train} & 14041\\\hline
	\texttt{Dev} & 3250\\\hline
	\texttt{Test} & 3453\\\hline
	\end{tabular}
	\caption{Data Split}
	\label{tab:conll_dataset_split}
\end{subtable}
\caption{CoNLL 2003 Dataset Stats}
\end{table}

\subsection{OntoNotes 5.0 Dataset}

\texttt{OntoNotes 5.0}\cite{} is a large corpus consisting of text from news reports, weblogs, broadcasts etc. labelled with structural syntactic and linguistic information like ontology, coreference etc. We focus on named entity tags from English language corpus. The dataset consists of 18 entity classes including \texttt{PERSON}, \texttt{FACILITY}, \texttt{PRODUCT} etc. along with entities with numerical semantics like \texttt{DATE}, \texttt{MONEY} etc. More details on what each entity means can be found in release docs\footnote{Section 2.6 of https://catalog.ldc.upenn.edu/docs/LDC2013T19/OntoNotes-Release-5.0.pdf}. We obtain the dataset available on GitHub\footnote{https://github.com/yuchenlin/OntoNotes-5.0-NER-BIO}. Entities like \texttt{GPE} (Geo-Political Entity) and \texttt{LOCATION} have very subtle differences. There are some similarities among other entities as well. Alphanumeric entity tokens constitute \texttt{1.04\%} of all entity tokens. The entities are flatly labelled (no nesting). As per Table \ref{tab:onto_entity_distribution}, entities like \texttt{PERSON}, \texttt{DATE} are high-resource where as \texttt{LANGUAGE}, \texttt{LAW} are low-resource. Average sentence length is \texttt{19.04} tokens. Table \ref{tab:onto_dataset_split} shows the \texttt{Train}/\texttt{Dev}/\texttt{Test} split. 

\begin{table}[h!]
\begin{subtable}[t]{.48\linewidth}
\centering
\begin{tabular}{|c|c|}\hline
	\textbf{Entity} & \textbf{Count}\\\hline
	\texttt{DATE} & 23786\\\hline
    \texttt{MONEY} & 6425\\\hline
    \texttt{WORK\_OF\_ART} & 1650\\\hline
    \texttt{CARDINAL} & 13626\\\hline
    \texttt{ORG} & 29963\\\hline
    \texttt{PERSON} & 27332\\\hline
    \texttt{GPE} & 28133\\\hline
    \texttt{NORP} & 11608\\\hline
    \texttt{PERCENT} & 4866\\\hline
    \texttt{ORDINAL} & 2737\\\hline
    \texttt{TIME} & 2289\\\hline
    \texttt{LOC} & 2691\\\hline
    \texttt{PRODUCT} & 1296\\\hline
    \texttt{FAC} & 1440\\\hline
    \texttt{EVENT} & 1273\\\hline
    \texttt{QUANTITY} & 1583\\\hline
    \texttt{LANGUAGE} & 412\\\hline
    \texttt{LAW} & 568\\\hline
	\end{tabular}
	\caption{Entity Distribution}
	\label{tab:onto_entity_distribution}
% }
\end{subtable}
\begin{subtable}[t]{.48\linewidth}
\centering
\begin{tabular}{|c|c|}\hline
	\textbf{Split} & \textbf{\# Sentences}\\\hline
	\texttt{Train} & 115812\\\hline
	\texttt{Dev} & 15680\\\hline
	\texttt{Test} & 12217\\\hline
	\end{tabular}
	\caption{Data Split}
	\label{tab:onto_dataset_split}
\end{subtable}
\caption{OntoNotes 5.0 English NER Dataset Stats}
\end{table}

\subsection{JNLPBA Dataset}

The \texttt{JNLPBA}\cite{} dataset comes from GENIA version 3.2 corpus\cite{} which consists of abstracts taken from MEDLINE database. GENIA dataset consists of 36 entity classes. For preparing JNLPBA, some of these classes are combined to a higher-level entity class and some are ignored. In all, JNLPBA has 5 entity classes: \texttt{protein}, \texttt{DNA}, \texttt{RNA}, \texttt{cell\_line}, \texttt{cell\_type}. We obtain the dataset available on GitHub\footnote{https://github.com/cambridgeltl/MTL-Bioinformatics-2016/tree/master/data/JNLPBA}. Alphanumeric entity tokens constitute \texttt{3.73\%} of all entity tokens. The entities are flatly labelled (no nesting). As per Table \ref{tab:jnlpba_entity_distribution}, mostly the entities can be considered high-resource although representation of \texttt{RNA} is comparatively very less. Average sentence length is \texttt{26.50} tokens. Table \ref{tab:jnlpba_dataset_split} shows the \texttt{Train}/\texttt{Dev}/\texttt{Test} split. 

\begin{table}[h!]
\begin{subtable}[t]{.48\linewidth}
\centering
\begin{tabular}{|c|c|}\hline
	\textbf{Entity} & \textbf{Count}\\\hline
	\texttt{protein} & 35336\\\hline
    \texttt{DNA} & 10589\\\hline
    \texttt{cell\_type} & 8639\\\hline
    \texttt{cell\_line} & 4330\\\hline
    \texttt{RNA} & 1069\\\hline
	\end{tabular}
	\caption{Entity Distribution}
	\label{tab:jnlpba_entity_distribution}
% }
\end{subtable}
\begin{subtable}[t]{.48\linewidth}
\centering
\begin{tabular}{|c|c|}\hline
	\textbf{Split} & \textbf{\# Sentences}\\\hline
	\texttt{Train} & 16807\\\hline
	\texttt{Dev} & 1739\\\hline
	\texttt{Test} & 3856\\\hline
	\end{tabular}
	\caption{Data Split}
	\label{tab:jnlpba_dataset_split}
\end{subtable}
\caption{JNLPBA Dataset Stats}
\end{table}

\subsection{BIONLP13CG Dataset}

The \texttt{BIONLP13CG}\cite{} (Cancer Genetics) dataset comes from BioNLP Shared Task 2013. The text belongs to the theme of biological processes relating to the development and progression of cancer. It consists of 16 entity types with a mix of high-resource and low-resource ones. We obtain the dataset available on the shared task website\footnote{http://2013.bionlp-st.org/tasks/cancer-genetics} and process it into \texttt{tsv} format. For most part of NER study when using this dataset we focus on flat-annotated entity mentions and ignore the small percentage of nested entities both from training and evaluation. The flat-annotated corpus is consistent with the one avaiable on GitHub\footnote{https://github.com/cambridgeltl/MTL-Bioinformatics-2016/tree/master/data/BioNLP13CG-IOB}. Alphanumeric entity tokens constitute \texttt{3.08\%} of all entity tokens. There are some entity overlaps and around 1\% of entities are nested. Table \ref{tab:bio_entity_distribution} shows the representation of entity mentions in the complete dataset. Average sentence length is \texttt{27.57} tokens. Table \ref{tab:bio_dataset_split} shows the \texttt{Train}/\texttt{Dev}/\texttt{Test} split. 

\begin{table}[h!]
\begin{subtable}[t]{.48\linewidth}
\centering
\begin{tabular}{|c|c|}\hline
	\textbf{Entity} & \textbf{Count}\\\hline
	\texttt{Gene\_or\_gene\_product} & 7908\\\hline
    \texttt{Cancer} & 2582\\\hline
    \texttt{Cell} & 3492\\\hline
    \texttt{Organism} & 1715\\\hline
    \texttt{Simple\_chemical} & 2270\\\hline
    \texttt{Multi-tissue\_structure} & 857\\\hline
    \texttt{Organ} & 421\\\hline
    \texttt{Organism\_subdivision} & 98\\\hline
    \texttt{Tissue} & 587\\\hline
    \texttt{Immaterial\_anatomical\_entity} & 102\\\hline
    \texttt{Organism\_substance} & 283\\\hline
    \texttt{Cellular\_component} & 569\\\hline
    \texttt{Pathological\_formation} & 228\\\hline
    \texttt{Amino\_acid} & 135\\\hline
    \texttt{Anatomical\_system} & 41\\\hline
    \texttt{Developing\_anatomical\_structure} & 35\\\hline
	\end{tabular}
	\caption{Entity Distribution}
	\label{tab:bio_entity_distribution}
% }
\end{subtable}
\begin{subtable}[t]{.48\linewidth}
\centering
\begin{tabular}{|c|c|}\hline
	\textbf{Split} & \textbf{\# Sentences}\\\hline
	\texttt{Train} & 3033\\\hline
	\texttt{Dev} & 1003\\\hline
	\texttt{Test} & 1906\\\hline
	\end{tabular}
	\caption{Data Split}
	\label{tab:bio_dataset_split}
\end{subtable}
\caption{BIONLP13CG Dataset Stats}
\end{table}

\section{Experimental Setup}
% TODO: arorja

\chapter{Methodology}
\label{chp:methodology}
Concretely, in named entity recognition (NER), we are given an unlabeled raw text and the goal is to identify certain entities (sequence of tokens) of interest. In this study, we approach this task using machine learning in supervised learning setting. The idea is, we are given a set of labeled sentences with the required entities marked. We feed those samples to a machine learning system to learn from it. The system is then evaluated on a manually labeled test set and relevant evaluation metrics are reported. In this chapter, we will look at various ways of approaching NER along with detailed ablation studies and variants. Simultaneously, we compare our results with existing published state-of-the-art approaches on multiple datasets.

\section{Sequence Labeling}
Traditionally, the most intuitive way of approaching named entity recognition is as a sequence labeling task. An input sentence can be considered as a sequence of \texttt{n} tokens, fed to a neural network model. For each token, the model classifies it into an output class as per \texttt{BIO} tagging scheme. As the model backbone, we use \texttt{CNN-LSTM-CRF} and \texttt{BERT} architectures as described below.

\begin{itemize}
    \item \texttt{CNN-LSTM-CRF}: As proposed in \cite{ma2016end}, this is a popular NER model. The character-level CNN helps capture the intrinsic patterns and word-level semantics of individual tokens. These character-level embeddings are concatenated with word embeddings and fed to bidirectional LSTM\cite{} which helps capture the sentence grammar and token inter-dependencies. Finally, the CRF models the output tag sequence and makes sure that abrupt and unexpected tag transitions do not take place. 
    
    \item \texttt{BERT}: Proposed in \cite{devlin2018bert}, BERT is a bidirectional encoder implemented using the transformer architecture \cite{}. The input is a sentence, which is broken down into sub-words/sub-tokens. Multi-head attention and several layers of encoders are able to capture long-term relationships among tokens and semantics well. Finally, the model outputs contextualized embeddings for each sub-token in the sentence. Then we have a simple fully-connected layer followed by Softmax to classify each token into an output class. The model is optimized using cross-entropy loss.
\end{itemize}

\subsection{Experiment Details}
For English news corpora like \texttt{CoNLL 2003} and \texttt{OntoNotes 5.0}, we use \texttt{bert-base-uncased} model from \texttt{transformers}\footnote{https://huggingface.co/transformers} python package in \texttt{PyTorch}. For biomedical datasets, we use \texttt{BioBERT-Base}\footnote{https://github.com/dmis-lab/biobert#download} model which is specifically pretrained on biomedical data. For \texttt{CNN-LSTM-CRF} framework, as word embeddings we use the contextual embeddings from BERT/BioBERT model but freeze the BERT architecture for training. So, the trainable model architecture still remains a core \texttt{CNN-LSTM-CRF}. We take the mean of sub-token embeddings from BERT to get the embedding for the token.

\begin{table}[h!]
\centering
\begin{tabular}{|c|c|c|c|c|}\hline
	\textbf{Model} & \textbf{BIONLP13CG} & \textbf{JNLPBA} & \textbf{CoNLL 2003} & \textbf{OntoNotes 5.0}\\\hline
	\texttt{BERT-Freeze} & 75.42 & todo & todo & todo \\\hline
	\texttt{CNN-LSTM-CRF (BERT-Freeze)} & 84.1 & 76.2 & 91.6 & 86.4 \\\hline
	\texttt{BERT} & 85.99 & 74.35 & 91.36 & 83.39 \\\hline
	\end{tabular}
    \caption{Results: Sequence Labeling (Test set Micro-F1 in \%)}
    \label{tab:res_seq_labeling}
\end{table}

\subsection{Observations}
Based on results summarized in Table \ref{tab:res_seq_labeling}:
\begin{itemize}
    \item Pretrained BERT embeddings themselves capture linguistic semantics well as can be seen from \texttt{BERT-Freeze}.
    \item Fine-tuning the BERT model or using BERT embeddings with CNN-LSTM-CRF both work well and give almost comprable enhancements over the \texttt{BERT-Freeze} setup.
\end{itemize}

\section{Question Answering}

% QA3, QA4, Where
The NER task can also be modelled as a question answering problem where, we give a question to the model asking it to extract the entity of interest from the supplied text. 

\cite{li2019unified} and \cite{li2019dice} show the effectiveness of this setup using a simple BERT model architecture on multiple general English and Chinese news datasets. They make the model output candidate spans (start and end indices) where the entity in question is present. This setup has advantages over the naive sequence labeling setup since using this framework we can extract even nested or overlapping entities while sequence labeling can capture only flat non-overlapping entities. Since they output spans, so their classification layer does a $\mathcal{O}(n^2)$ computation where $n$ is the number of tokens in the input sentence. \cite{banerjee2019knowledge} use a similar setup and output simple \texttt{B}, \texttt{I} and \texttt{O} tags for each token to mark the presence of the entity in question. Hence, their problem complexity becomes $\mathcal{O}(n)$. On top of this setup, we study the variations described below.

\begin{itemize}
    \item \textbf{Tagging Scheme}: Classify each token into 3 output classes (\texttt{B}, \texttt{I} and \texttt{O}) [\texttt{BERT-QA(BIO)} model] or 4 classes explicitly modeling the end boundary using \texttt{E} output class [\texttt{BERT-QA(BIOE)} model]. In both these models, we ask questions of the form, \textit{What is the person mentioned in the text?}.
    
    \item \textbf{Question Formulation}: \cite{banerjee2019knowledge} show that their trained model gives a high importance to the question word. Hence we probe the model and change the question slightly to observe its impact. Instead of asking \textit{What} [\texttt{BERT-QA(What)} model], we ask \textit{What is the person mentioned in the text?} [\texttt{BERT-QA(Where)} model]. In both these models, we follow the \texttt{BIOE} output tagging scheme.
\end{itemize}

\subsection{Experiment Details}


\begin{table}[h!]
\centering
\begin{tabular}{|c|c|c|c|c|}\hline
	\textbf{Model} & \textbf{BIONLP13CG} & \textbf{JNLPBA} & \textbf{CoNLL 2003}\\\hline
	\texttt{BERT-QA(BIO)} & todo & 74.81 & todo\\\hline
	\texttt{BERT-QA(BIOE)} & 86.45 & todo & 91.17\\\hline
	\end{tabular}
    \caption{Results: Question Answering (Tagging Scheme) (Test set Micro-F1 in \%)}
    \label{tab:res_qa_tagging}
\end{table}

\begin{table}[h!]
\centering
\begin{tabular}{|c|c|c|c|}\hline
	\textbf{Model} & \textbf{BIONLP13CG} & \textbf{JNLPBA} & \textbf{CoNLL 2003}\\\hline
	\texttt{BERT-QA(What)} & 86.45 & todo & 91.17\\\hline
	\texttt{BERT-QA(Where)} & \textbf{86.83} & 74.64 & \textbf{91.82}\\\hline
	\end{tabular}
    \caption{Results: Question Answering (Question Formulation) (Test set Micro-F1 in \%)}
    \label{tab:res_qa_question}
\end{table}

\subsection{Observations}
\begin{itemize}
    \item From Table \ref{tab:res_qa_tagging}, \texttt{BIOE} tagging is able to better capture the entity boundaries as compared to \texttt{BIO} tagging scheme.
    
    \item From Table \ref{tab:res_qa_question}, we observe that the model is sensitive to slight changes in question semantics. Asking a \textit{where} to the model helps it learn and identify entities better than asking \textit{what}. 
\end{itemize}

\section{Span Detection and Classification Pipeline}
Another way of approaching NER is to break it down into a two-step pipelined procedure. In the first step, given a sentence, we detect all entity spans (Span Detector). In the next step, these spans are classified into an output entity class by another model (Span Classifier).

\subsection{Experiment Details}

\begin{itemize}
    \item \textbf{Span Detection}: We treat this sub-problem as a question answering task. Every sample sentence of the form, \textit{Emily}[\texttt{PERSON}] \textit{lives in United States}[\texttt{LOCATION}], is converted into the form, \textit{What is the entity mentioned in the text? Emily lives in United States}. This is fed to BERT (\texttt{bert-base-uncased}) model and the outputs are passed through a single fully connected layer followed by softmax. The model is expected to output \texttt{B}, \texttt{I} and \texttt{O} tags for each token and in this case, detect two spans, \textit{Emily} and \textit{United States}. 
    
    \item \textbf{Span Classification}: Again, we treat this sub-problem as a question answering task as well. For every gold labeled entity mention in a training set sentence, \textit{Emily}[\texttt{PERSON}] \textit{lives in United States}[\texttt{LOCATION}], we form a sample, \textit{Emily lives in United States. What is Emily?} The sentence is passed to BERT (\texttt{bert-base-uncased}) and the final pooled output for the sentence is fed to a fully connected layer followed by softmax to classify the sentence into one of the entity types, in this case, \texttt{PERSON}.
    
    \item \textbf{Pipeline}: During evaluation, every unlabeled sentence is passed through Span Detector and for each output span, we convert to an input sample for Span Classifier.
\end{itemize}

\begin{table}[h!]
\centering
\begin{tabular}{|c|c|c|c|c|}\hline
	\textbf{} & \textbf{BIONLP13CG} & \textbf{JNLPBA} & \textbf{CoNLL 2003}\\\hline
	\texttt{Span Detection} & 90.12 & 78.35 & 95.23\\\hline
	\texttt{Span Classification} & 94.06 & 95.08 & 94.50\\\hline
	\texttt{Pipeline} & 85.89 & 75.01 & 91.64\\\hline
	\texttt{BERT-QA} & 86.45 & todo & 91.17\\\hline
	\end{tabular}
    \caption{Results: Span Pipeline (Test set Micro-F1 in \%)}
    \label{tab:res_span}
\end{table}

\subsection{Observations}
Table \ref{tab:res_span} reports the results of the pipelined span detection and classification procedure and also presents comparison with simple BERT question answering setup for NER. We present this comparison since question answering model serves as the primary backbone of our current span-based extraction procedure. Note that for one-to-one comparison all results here correspond to \texttt{B}, \texttt{I}, \texttt{O} output tagging and \textit{What} as question word in question formulation.

\begin{itemize}
    \item \textbf{Span Detection}: Detecting all entity spans together without classification is a simpler problem for the model than full NER and hence we get better performance compared to \texttt{BERT-QA} model.
    
    \item \textbf{Span Classification}: Given that spans are detected correctly, this second step of the pipeline is relatively simple for the BERT model. On all datasets, we see above 90\% Micro-F1 on test set.
    
    \item \textbf{Pipeline}: The pipelined procedure gives comparable results to standard question answering based NER model. The main bottleneck lies in the span detection part. Since this procedure is pipelined, errors in first step propagate to the next step leading to an overall reduced performance.
\end{itemize}

\section{Learning Objective Variations}
An ML algorithm learns by optimizing its learning objective. For named entity recognition in supervised learning setup, we get a corpus labeled with gold entity mentions. As highlighted in section \ref{sec:nature_of_entities}, there may be an inherent labeling bias in the dataset. Some entities have more representation, more labeled mentions (\textit{high-resource}) while others are rare entities (\textit{low-resource}). With fewer samples, it becomes difficult for the model to learn good differentiating rules for extracting the low-resource entities. To handle this bias in dataset, it is common to give more importance/weight to low-resource entity samples in calculating the learning objective and optimizing on it. In this direction, we experiment with the variants detailed below.

\begin{itemize}
    \item \textbf{Cross-Entropy (CE) Loss}: This is the standard objective function in which we calculate the cross entropy between the predicted and gold labels for each token in the sentence.
    
    \item \textbf{Weighted Cross-Entropy Loss}: Here, based on the gold-labeled tag for a token we assign a weight to its contribution. All tokens which belong to a valid entity span contribute equally to the loss and all other tokens contribute with a reduced weight of 0.5. This helps because a large part of the corpus consists of tokens which belong to the \texttt{O} tag category. For instance, the \texttt{BioNLP13CG} corpus has 76.5\% tokens with \texttt{O} tag.
    
    \item \textbf{Punctuation Weighted CE Loss}: From qualitative analysis of misclassified samples in standard CE loss setup we noticed that the model is not able to learn good representations for special symbols like parenthesis, hyphen, period, slash etc. Hence, we force the model to learn these symbols well by penalizing the model twice if the misclassified token is a punctuation/special symbol.
    
    \item \textbf{Dice Loss}: As proposed by \cite{li2019dice}, we address the above mentioned data imbalance issue between entity and non-entity tokens using dice loss, based on S{\o}rensen-Dice coefficient\cite{} or Tversky index\cite{}. This helps because standard CE loss in accuracy-oriented while during evaluation we calculate F1-measure. The dice loss gives equal importance to false-positives and false-negatives at training time and hence reduces the discrepancy among these training and test time metrics.
    
    \item \textbf{CRF}: Apart from the cross-entropy objective, we also experiment by adding a CRF layer on the fully connected layer output in standard BERT-based model. CRF layer is found to work well with bidirectional LSTM\cite{ma2016end} in modeling output tag transitions and emissions.
    
\end{itemize}

\subsection{Experiment Details}
For CRF implementation, we use \texttt{torchcrf} python package.

\begin{table}[h!]
\centering
\begin{tabular}{|c|c|c|c|c|}\hline
	\textbf{} & \textbf{BIONLP13CG} & \textbf{CoNLL 2003}\\\hline
	\texttt{CE Loss} & 85.99 & 91.36\\\hline
	\texttt{Weighted CE Loss} & 85.93 & todo\\\hline
	\texttt{Punctuation CE Loss} & 86.12 & todo\\\hline
	\texttt{Dice Loss} & 86.35 & 90.76\\\hline
	\texttt{CRF} & 86.20 & todo\\\hline
	\end{tabular}
    \caption{Results: Learning Objectives (Test set Micro-F1 in \%)}
    \label{tab:res_loss}
\end{table}

\subsection{Observations}
\begin{itemize}
    \item \texttt{Weighted CE Loss} is found to perform comparable to \texttt{CE Loss} and both do not perform as well as the other variants.
    
    \item In \texttt{BioNLP13CG} corpus, \texttt{Dice Loss} performs well as this corpus as several high and low resource entities (data imbalance). \texttt{Dice Loss} is not able to give its advantages with \texttt{CoNLL 2003} corpus since here all 4 entity types have a comparable and high representation.
    
    \item \texttt{Punctuation CE Loss} is able to make the model learn slightly better semantics of special symbols and hence performs better than the standard \texttt{CE Loss} counterpart.
    
    \item \texttt{CRF} is able to capture the output tag transitions better and hence performs better than standard \texttt{CE Loss} setting.
\end{itemize}

\section{Capturing Additional Token Semantics}
\label{sec:additional_token_semantics}

Both in \texttt{CNN-LSTM-CRF} and transformer-based architectures, we rely on on a pretrained BERT model. For biomedical datasets we use BioBERT which is trained on scientific and biomedical literatures while for English news datasets, we work with \texttt{bert-base-uncased} model which is trained on Wikipedia and books. BERT models rely on Word-Piece tokenizer\cite{} which considers breaking words into sub-words for representation. Even then from qualitative analysis we find that there are several terms whose semantics are not captured well by the existing BERT model. This causes errors which can be categorized as:

\begin{itemize}
    \item \textbf{Out of Vocabulary Tokens}: News articles and scientific texts both sometimes make use of abbreviations or localized entities which may be specific to that news article, event time or research paper, but are rare otherwise. Additionally biomedical texts also consist of chemical names in scientific form which includes numerals etc. which need to be extracted. Semantics of such terms is not captured well by generically pretrained models. Table \ref{tab:oov_issue} shows some examples from \texttt{BioNLP13CG} corpus.
    
    \begin{table}[h!]
    \centering
    \begin{tabular}{|c|c|}\hline
    	\textbf{Entity} & \textbf{Misclassification Examples}\\\hline
    	\texttt{Gene\_or\_Gene\_Product} & DPD, Xhol, mutCK1delta, FAS\\\hline
    	\texttt{Simple\_Chemical} & MnCl2, AglRhz, NO\\\hline
    	\texttt{Cell} & LoVo, DeltaG45, BMSVTs\\\hline
    	\texttt{Amino\_Acid} & phosphoS727, Y705F\\\hline
    	\end{tabular}
        \caption{Out-of-Vocabulary tokens in \texttt{BioNLP13CG} corpus}
        \label{tab:oov_issue}
    \end{table}
    
    \item \textbf{Special Symbols}: Several entities to be extracted have hyphens, periods, parenthesis within them which are not captured well by pretrained BERT model leading to partial entity detection. Table \ref{tab:boundary_issue} gives some examples from \texttt{BioNLP13CG} corpus.
    
    \item \textbf{Modifier Suffix/Prefix}: Apart from the root entity required to be extracted the gold labels sometimes expect a modifier term as well which occurs as a prefix/suffix. Missing these again leads to boundary detection issues. Table \ref{tab:boundary_issue} gives some examples from \texttt{BioNLP13CG} corpus.
    
    \begin{table}[h!]
    \centering
    \begin{tabular}{|c|c|c|}\hline
    	\textbf{Misclassification Category} & \textbf{Gold} & \textbf{Predicted}\\\hline
    	\texttt{Special Symbols} & L . Se ( + ) cells & L . Se\\\hline
    	\texttt{Special Symbols} & Gs - IB ( 4 - ) ion & Gs - IB ( 4\\\hline
    	\texttt{Modifier Suffix/Prefix} & epicardial coronary artery & coronary artery\\\hline
    	\texttt{Modifier Suffix/Prefix} & T140 analogs & T140\\\hline
    	\end{tabular}
        \caption{Boundary detection issues in \texttt{BioNLP13CG} corpus}
        \label{tab:boundary_issue}
    \end{table}
\end{itemize}

\subsection{Experiment Details}

To address the above mentioned issues, we provide additional inputs and infrastructure to the model to learn the underlying semantics better.

\begin{itemize}
    \item \textbf{Special Symbol Features}: Before feeding the output of BERT model to final fully-connected classification layer, we add an extra one-hot dimension which is set if the current input token is a pure special symbol. This means we assign \texttt{1} for \textit{hyphen}(\texttt{-}), \textit{parenthesis}(\texttt{(} and \texttt{)}), \textit{comma}(\texttt{,}) etc. while assign \texttt{0} for tokens like \texttt{carbon}, \texttt{123}, or mixed format tokens like \texttt{Ca(2+)}, \texttt{AB-3} etc.
    
    \item \textbf{Word Type Features}: As an extension of the above special symbol features, here we associate each token with a word type as shown in Table \ref{tab:word_type_encoding} which is converted into a one-hot vector concatenated with BERT output embeddings before sending to final classifier layer.
    
    \item \textbf{Character and Pattern Features}: Chemical formulas and scientific names generally follow a nomenclature convention or pattern. Similarly, out-of-vocabulary tokens also may have some intrinsic character level information which is not well captured by the BERT model which considers sub-words. Infact this issue with BERT is studied well in \cite{boukkouri2020characterbert} which propose to use a character CNN instead of word-piece tokenizer at the stating stage. Motivated by \texttt{CNN-LSTM-CRF} setup and this study, we do the following:
    
    \textbf{Modeling characters}. Each word is passed to BERT and in parallel, to five 1-dimensional CNNs with kernel sizes of 1 to 5, each having 16 input and 16 output channels. Input character is indexed and embedded into 16-dimensions through an embedding layer. The CNN outputs are concatenated and passed through a linear layer to get overall 768-dimensional output vector for each token.
    
    \textbf{Modeling patterns}. Each word is converted to a pattern (like regular expression, a denser space than simply all characters) converting all uppercase letters to \texttt{U}, lowecase to \texttt{L}, digits to \texttt{D} etc. and then sent to a separate character CNN (like the one described above) and then to a bidirectional LSTM to get contextual token embeddings.
    
    Finally, these character and pattern embeddings are concatenated with BERT outputs and fed to final classifier layer for tag classification.
    
    \item \textbf{Part-of-Speech and Dependency Parse Features}: Concatenate BERT embeddings with the one-hot POS and one-hot dependency parse features before feeding to final classifier layer.
    
    \item \textbf{Head Tokens}: BERT uses word-piece tokenizer and hence can break a given single token into multiple sub-words. Instead of doing a token classification on each of these sub-words and making sure everything is correct, it is simpler to take the embedding output for the first sub-word (\textit{head}) for each token. This technique is also used in the original BERT paper \cite{devlin2018bert} for their NER experiments. 
    
    \item \textbf{Highway Network}: Instead of using a simple concatenation of character features with BERT as done in previous experiment, here we create a highway network\cite{}, similar to the one used in BiDAF\cite{} architecture in which we train a gated network to learn when to use BERT vectors and when to use the additional character-level information.
\end{itemize}

\subsection{Observations}


\section{Training Effectiveness Study}
In the previous section \ref{sec:additional_token_semantics}, we mostly try to give additional information to a BERT architecture by concatenating BERT outputs with some additional feature vectors before final classification. Giving additional information to a model is one thing but whether the model is actually able to pick cues from the additional information or not, is another. In this section, we study the training effectiveness on \texttt{BioNLP13CG} dataset in \texttt{Sequence Labeling} setting using \texttt{BioBERT-Base} model.

\subsection{Feeding Answer as Input}

To study the training effectiveness, we give the model the best ideal-case information, i.e. for each token, we feed its gold label as a one-hot vector. This is concatenated with BERT output before feeding to final classifier. We study the following variants:

\begin{itemize}
    \item \texttt{BERT(Freeze) + Answer}: We freeze the BERT model parameters and concatenate the answer vectors. This effectively reduces the no. of trainable parameters to a few thousand. We try this in two settings, low learning rate of $10^{-5}$ (recommended BioBERT learning rate\footnote{https://github.com/dmis-lab/biobert#named-entity-recognition-ner} and high learning rate of $0.005$.
    
    \item \texttt{BERT + Answer}: This mimics the real-world scenario where the BERT model is fine-tuned with given additional information at a low learning rate of $10^{-5}$.
\end{itemize}

From the results summarized in Table \ref{}, we observe that with a very low learning rate of $10^{-5}$, the model is not able to learn from the additional answer information provided (both with frozen BERT and otherwise). At a higher learning rate of $0.005$, the model catches the provided cue well.

\subsection{What happens at high learning rate?}

From the previous section, we see that we need a relatively higher learning rate to learn from additional information fed to the model. But with BERT fine-tuning, it is recommended to use a very low learning rate in the order of $10^{-5}$. To study this, we pass answer spans as input to the model. Basically, for each token, we add a one-hot dimension which is set if the token belongs to a valid entity else it is set to $0$. This means the model is actually being fed the gold spans and all it has to do is span classification. This should be relatively easy and give good results as seen previously in Section \ref{}. We experiment with the following variants:

\begin{itemize}
    \item \texttt{BERT (Freeze)}: Freezing BERT model and passing through a single trainable classifier layer. This model has only around $26,000$ trainable parameters. 
    
    \item \texttt{BERT (Freeze) + Gold Span}: Same as the above but with gold spans given as additional input. So, this model is expected to perform better than \texttt{BERT (Freeze)}.
    
    \item \texttt{BERT + Gold Span}: Same as the above model but with trainable BERT model. We try training this in two different settings, with low learning rate of $10^{-5}$ and high learning rate of $0.005$.
    
    \item \texttt{BERT}: This is the standard BERT model with final classification layer and no additional feature inputs fine-tuned at learning rate of $10^{-5}$.
\end{itemize}

From Table \ref{}, we make the following observations:

\begin{itemize}
    \item From \texttt{BERT (Freeze)} and \texttt{BERT (Freeze) + Gold Span}, we observe that indeed giving gold span information helps the model.
    
    \item From \texttt{BERT(Freeze) + Gold Span} and \texttt{BERT + Gold Span (LR: $10^{-5}$)}, we observe that indeed fine-tuning is able to learn the semantics of entities much better than when using the BERT embeddings right out-of-the-box.
    
    \item However, from \texttt{BERT} and \texttt{BERT + Gold Span (LR: $10^{-5}$)}, we observe that with a low learning rate, the model is not able to focus on and effectively utilize the gold span information.
    
    \item Finally, from \texttt{BERT + Gold Span (LR: $10^{-5}$)} and \texttt{BERT + Gold Span (LR: $0.005$)}, we observe that increasing the learning rate has a deteriorating effect on the pretrained BERT parameters and the rigorous push from high learning rate pushes the model to an unsatisfactory local optima.
\end{itemize}

\section{Tagging Scheme Variation}
As described in Section \ref{sec:tagging_scheme}, in this section, we study how much impact do different output tagging schemes have on boundary detection and learning of the model. We test the \texttt{2-Tag}, \texttt{BIO} and \texttt{BIOE} tagging schemes in question answering setup on \texttt{BIONLP13CG} corpus.

We omit the experiments with sequence labeling setup since for \texttt{K} output tags, \texttt{2-Tag} scheme gives \texttt{K + 1} output tags (\texttt{+1} for \texttt{O} tag). With \texttt{BIO} scheme, we have \texttt{2K + 1} output tags and with \texttt{BIOE} scheme, we have \texttt{3K + 1} tags. For \texttt{BIONLP13CG} corpus which already has \texttt{K = 16}, the \texttt{BIOE} scheme makes number of output tags as \texttt{49}, which is too large to train well since we don't have enough training data. However, for the question answering setup, the number of output tags remains \texttt{3} for \texttt{2-Tag}, \texttt{4} for \texttt{BIO} and \texttt{5} for \texttt{BIOE} scheme which is manageable.

From results in Table \ref{}, we observe that as expected, explicitly modeling the begin and end boundaries performs the best while not modeling start and end at all, in \texttt{2-Tag} scheme performs the least among them. However this will have lesser number of parameters to train and more representative samples for each case.

\section{Pretrained Model Variation}


\section{Clustering and Segregation of Diverse Entities}
In this section, we focus on a dataset with known data imbalance issues (presence of high and low-resource entities) like \texttt{BIONLP13CG} dataset.

\section{Nested Entities and Supplying Additional Context}

\section{Comparative Precision/Recall Analysis}

\chapter{Related Work}
\label{chp:related_work}
In the past few years, deep learning approaches have been increasingly applied for NER \cite{torfi2020natural, li2020survey}, a popular architecture being CNN-LSTM-CRF\cite{ma2016end}. 
% Training complex deep learning systems have however been limited by the shortage of labeled training data. Nevertheless, there have been efforts to generate noisy labeled data as weak supervision signals for training. \cite{shang2018learning} propose AutoNER framework which uses distant supervision, \cite{arora2017extracting} use regular expression patterns for artificial training data generation, \cite{zhou2019dual} propose adversarial perturbations and \cite{liu2018empower, liu2018efficient} train a task-aware language model from unlabeled data which guides NER. 
Recently with the advent of BERT\cite{devlin2019bert}, it has become possible to transfer knowledge from massive pretrained language models and fine-tune them giving good performance even on small datasets.

\cite{xu2021better} propose a Syn-LSTM setup leveraging dependency tree structure along with pretrained BERT embeddings for NER. \cite{li2020MRC} and \cite{li2019dice} have a system very similar to our system and use QA-based setup over BERT for NER. However, they enumerate all possible span start and end locations in a sentence leading to an $\mathcal{O}(n^2)$ complexity where $n$ denotes the sentence size. Our architecture design does a token level classification and hence has $\mathcal{O}(n)$ complexity. Recently, \cite{yan2021unified} propose a generative framework leveraging BART\cite{lewis2019bart} for NER. \cite{yu2020named} propose a biaffine model utilizing pretrained BERT and FastText \cite{bojanowski2017enriching} embeddings along with character-level CNN setup over a bi-LSTM architecture. All of these models leverage BERT in some way or the other and report good performance on \data{OntoNotes 5.0} corpus however all of them make use of BERT-Large architecture. 

\cite{wang2021improving} give good performance on \data{WNUT17} dataset by leveraging external knowledge and a cooperative learning setup. Recently, \cite{nguyen2020bertweet} propose a new BERT model, BERTweet trained on English tweets. Currently, we use the standard pretrained BERT model by \cite{devlin2019bert} on \data{WNUT17}. Our framework can easily be trained using BERTweet model as the backbone and this can enhance our results on \data{WNUT17} even further.

In scientific and biomedical domain, NER has its own set of challenges. Entity mentions are long and may have alpha-numeric symbols and chemical formulas which may be hard for a language model to make sense of. On \data{BioNLP13CG} corpus, \cite{crichton2017neural} report $78.90$ as test set micro-F1 in a multi-task learning setup and \cite{neumann2019scispacy} report $77.60$ using their SciSpacy system. BioBERT\cite{lee2020biobert} authors does not state their performance on this dataset however \cite{banerjee2019knowledge} replicate the BioBERT model as a baseline and report $85.56$. Other works like \cite{wang2019cross} use a multi-task learning framework and combine labelled data from multiple corpora mapping entity tags across corpora to coherent classes. \cite{wang2019distantly} make use of a setup similar to AutoNER\cite{shang2018learning} tailor-made for biomedical NER. 


\chapter{Conclusions and Future Work}
\label{chp:conclusions}
In this work, we looked at the NER problem from three different perspectives, namely, sequence labeling, question answering and span-based approach. We compared and contrasted them with each other and studied their advantages and limitations. Taking inspiration from the QA setup, we proposed the span detection and classification pipeline which uses a reverse question formulation. Additionally, we also proposed to convert from a sparse character space to a dense pattern space through which we can learn meaningful intrinsic character patterns in alphanumeric and pattern-oriented entities. We demonstrated the effectiveness of our proposed domain-agnostic techniques on multiple datasets in general English and biomedical domains. We also presented a study depicting that trivial concatenation of external semantic vectors with BERT outputs may not train the model effectively at lower learning rates.

Our span-based setup opens up prospects for more intuitive and creative ways of approaching the NER problem. However, the pipelined nature of the approach currently serves as a bottleneck. It may be worthwhile to think of some ensemble-based approach where we train individual BERT models on some sub-problems and each of those models contributed its part to solve the overall NER problem in a majority-voting setup.

Our study on training effectiveness reveals that feeding additional external semantics while fine-tuning the BERT model is non-trivial. This again motivates future research on designing feature fusion techniques which are effective with a BERT (transformer-like) architecture.

From the qualitative analysis of the various approaches, we observe that boundary detection serves as a primary issue in NER. To alleviate this problem, we explicitly model word types and special symbols. However, there is still a wide margin to cover. We encourage the research community to design architectures or new training objectives tailor-made to handle mention boundaries effectively. 

% NOTE 1: The Graduate College standards allow sections to be numbered by chapter number, section number, and subsection number. This means you can use the following commands within a chapter:
% \section{}
% \subsection{}
% \paragraph{} (does not produce a number)
% In other words, do not use the command \subsubsection{} and beyond!

% NOTE 2: The Graduate College is picky about access white space, so you should attempt to minimize access whitespace when possible. For example, Latex will move a section header and subsequent paragraph onto the next page to avoid having a section header followed by a single line of text. In this case, you should use the command "\clearpage \noindent" at the end of the first line text in the paragraph to try to bump the header and one line of text back onto the previous page.

%\chapter{Conclusions}
%\label{chp:concl}
%In this work, we looked at the NER problem from three different perspectives, namely, sequence labeling, question answering and span-based approach. We compared and contrasted them with each other and studied their advantages and limitations. Taking inspiration from the QA setup, we proposed the span detection and classification pipeline which uses a reverse question formulation. Additionally, we also proposed to convert from a sparse character space to a dense pattern space through which we can learn meaningful intrinsic character patterns in alphanumeric and pattern-oriented entities. We demonstrated the effectiveness of our proposed domain-agnostic techniques on multiple datasets in general English and biomedical domains. We also presented a study depicting that trivial concatenation of external semantic vectors with BERT outputs may not train the model effectively at lower learning rates.

Our span-based setup opens up prospects for more intuitive and creative ways of approaching the NER problem. However, the pipelined nature of the approach currently serves as a bottleneck. It may be worthwhile to think of some ensemble-based approach where we train individual BERT models on some sub-problems and each of those models contributed its part to solve the overall NER problem in a majority-voting setup.

Our study on training effectiveness reveals that feeding additional external semantics while fine-tuning the BERT model is non-trivial. This again motivates future research on designing feature fusion techniques which are effective with a BERT (transformer-like) architecture.

From the qualitative analysis of the various approaches, we observe that boundary detection serves as a primary issue in NER. To alleviate this problem, we explicitly model word types and special symbols. However, there is still a wide margin to cover. We encourage the research community to design architectures or new training objectives tailor-made to handle mention boundaries effectively.   % Inserts content from "conclusions.tex" here

%%%%%%%%%%%%%%%%%%%%%%%%%%%%%%%%%%%%%%%%%%%%%%%%%%%%%%%%%%%%%%%%%%%%%%%%%%%%%%%
% BIBLIOGRAPHY
%
\bibliographystyle{IEEE_ECE}
\bibliography{thesisrefs}  % Put references in BibTeX format in thesisrefs.bib.

%%%%%%%%%%%%%%%%%%%%%%%%%%%%%%%%%%%%%%%%%%%%%%%%%%%%%%%%%%%%%%%%%%%%%%%%%%%%%%%
% APPENDIX
%
% NOTE: Appendices go *after* the bibliography (see here: https://grad.illinois.edu/thesis/format). However, if appendices contain citations, then you may move the appendices *before* the bibliography section.
\appendix

%\chapter{Something}
%\label{apx:something}
%\input{appendix-something}  % inserts content from "appendix-name.tex"

\backmatter

\end{document}
\endinput
